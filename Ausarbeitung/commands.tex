% draws an arrow from (#2,#3) to (#2+#4,#3) with height of #5 (arrow in x direction)
% color (and other arguments, like visible on), startX, startY, length (X dir), height (Y dir)
\newcommand{\TikZArrowX}[5]{
    \filldraw[#1] (#2,#3) -- (#2,#3+#5*2/3) -- (#2+#4/2,#3+#5*2/3) -- (#2+#4/2,#3+#5) -- (#2+#4,#3) -- (#2+#4/2,#3-#5) -- (#2+#4/2,#3-#5*2/3) -- (#2,#3-#5*2/3) -- (#2,#3);
    \draw[#1, black] (#2,#3) -- (#2,#3+#5*2/3) -- (#2+#4/2,#3+#5*2/3) -- (#2+#4/2,#3+#5) -- (#2+#4,#3) -- (#2+#4/2,#3-#5) -- (#2+#4/2,#3-#5*2/3) -- (#2,#3-#5*2/3) -- (#2,#3);
}

% draws an arrow from #2,#3) to (#2,#3+#5) with height of #4 (arrow in y direction)
% color (and other arguments, like visible on), startX, startY, length (X dir), height (Y dir), color
\newcommand{\TikZArrowY}[5]{
    \filldraw[#1] (#2,#3) -- (#2+#4*2/3,#3) -- (#2+#4*2/3,#3+#5/2) -- (#2+#4,#3+#5/2) -- (#2,#3+#5) -- (#2-#4,#3+#5/2) -- (#2-#4*2/3,#3+#5/2) -- (#2-#4*2/3,#3) -- (#2,#3);
    \draw[#1, black] (#2,#3) -- (#2+#4*2/3,#3) -- (#2+#4*2/3,#3+#5/2) -- (#2+#4,#3+#5/2) -- (#2,#3+#5) -- (#2-#4,#3+#5/2) -- (#2-#4*2/3,#3+#5/2) -- (#2-#4*2/3,#3) -- (#2,#3);
}

% draws one entry of the color legend for the program scheme
% color, text, posX, posY (left, top coordinate)
\newcommand{\colorLegendEntry}[4]{
    \filldraw[#1] (#3,#4) rectangle (#3+1,#4+1);
    \draw[black] (#3,#4) rectangle (#3+1,#4+1);
    \node at (#3+0.5,#4+0.4) (color_legend_entry_point) {};
    \node[right=2mm of color_legend_entry_point] (color_legend_entry_text) {#2};
}

% draws a color legend for the program scheme
% posX, posY (left, top coordinate)
\newcommand{\colorLegend}[2] {
    \colorLegendEntry{black!30!orange}{Static translator parts}{#1}{#2}
    \colorLegendEntry{blue}{Intermediate representation (IR)}{#1}{#2-1.5}
    \colorLegendEntry{purple}{Machine code / ELF file}{#1+20}{#2}
}

% draws the schematic presentation of the program
% scale, fontSize
\newcommand{\ProgramScheme}[2]{
    \begin{tikzpicture}[scale=#1]
        % draw color legend
        \colorLegend{-1}{-6}

        % translator rectangle
        \draw[black, thick] (2.5,4) rectangle (27.5, -4);
        \node[align=center, color=black] at (15, 3) (translator_text) {\fontsize{#2}{#2} \selectfont Translator};

        % riscv elf file rectangle
        \draw[black, fill=purple] (-2,-2) rectangle (2,2);
        % label
        \node[align=center, color=white] at (0,0) (riscv_text) {\fontsize{#2}{#2} \selectfont RISC-V \\ \fontsize{#2}{#2} \selectfont Binary};

        % lifter arrow
        \TikZArrowX{black!30!orange}{2}{0}{6}{1.5}
        % arrow label
        \node[align=center, color=white] at (4.5,0) (lifter_text) {\fontsize{#2}{#2} \selectfont Lifter};

        % ir (unoptimized) rectangle
        \draw[black, fill=blue] (8,-2) rectangle (12,2);
        % label
        \node[align=center, color=white] at (10,0) (ir_text_1) {\fontsize{#2}{#2} \selectfont IR};

        % optimizer arrow
        \TikZArrowX{black!30!orange}{12}{0}{6}{1.5}
        % arrow label
        \node[align=center, color=white] at (14.5,0) (optimizer_text) {\fontsize{#2}{#2} \selectfont Optimizer};

        % ir (optimized) rectangle
        \draw[black, fill=blue] (18,-2) rectangle (22, 2);
        % label
        \node[align=center, color=white] at (20,0) (ir_text_2) {\fontsize{#2}{#2} \selectfont IR};

        % compiler arrow
        \TikZArrowX{black!30!orange}{22}{0}{6}{1.5}
        % arrow label
        \node[align=center, color=white] at (24.5,0) (generator_text) {\fontsize{#2}{#2} \selectfont Generator};

        % x86_64 rectangle
        \draw[black, fill=purple] (28.5,-2) rectangle (32.5,2);
        % label
        \node[align=center, color=white] at (30.5,0) (x86_64_text) {\fontsize{#2}{#2} \selectfont Trans- \\ \fontsize{#2}{#2} \selectfont lated \\ \fontsize{#2}{#2} \selectfont Binary};

        % helper rectangle
        \draw[black, fill=purple] (28.5, 9) rectangle (32.5, 5);
        % helper label
        \node[align=center, color=white] at (30.5, 7) (helper_text) {\fontsize{#2}{#2} \selectfont Helper \\ \fontsize{#2}{#2} \selectfont Library};

        % connection between x86_64 and helper library
        \draw[black, -{Stealth[scale=3]}] (30, 2) -- (30, 5);
        \draw[black, {Stealth[scale=3]}-] (31, 2) -- (31, 5);

        % result binary rectangle
        \draw[black, thick] (28, 12) rectangle (33, -4);
        % result binary label
        \node[align=center, color=black] at (30.5, 10.5) (result_binary_text) {\fontsize{#2}{#2} \selectfont Result \\ \fontsize{#2}{#2} \selectfont Binary};
        
        % memory image arrow + label
        \draw[black, -{Stealth[scale=3]}] (0, 2) -- (0, 7) -- (28, 7);
        \node[align=center, color=black] at (15, 8) {\fontsize{#2}{#2} \selectfont Memory Image};
    \end{tikzpicture}
}
