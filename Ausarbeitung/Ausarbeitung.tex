% Diese Zeile bitte -nicht- aendern.
\documentclass[course=eragp]{aspdoc}

% draws an arrow from (#2,#3) to (#2+#4,#3) with height of #5 (arrow in x direction)
% color (and other arguments, like visible on), startX, startY, length (X dir), height (Y dir)
\newcommand{\TikZArrowX}[5]{
    \filldraw[#1] (#2,#3) -- (#2,#3+#5*2/3) -- (#2+#4/2,#3+#5*2/3) -- (#2+#4/2,#3+#5) -- (#2+#4,#3) -- (#2+#4/2,#3-#5) -- (#2+#4/2,#3-#5*2/3) -- (#2,#3-#5*2/3) -- (#2,#3);
    \draw[#1, black] (#2,#3) -- (#2,#3+#5*2/3) -- (#2+#4/2,#3+#5*2/3) -- (#2+#4/2,#3+#5) -- (#2+#4,#3) -- (#2+#4/2,#3-#5) -- (#2+#4/2,#3-#5*2/3) -- (#2,#3-#5*2/3) -- (#2,#3);
}

% draws an arrow from #2,#3) to (#2,#3+#5) with height of #4 (arrow in y direction)
% color (and other arguments, like visible on), startX, startY, length (X dir), height (Y dir), color
\newcommand{\TikZArrowY}[5]{
    \filldraw[#1] (#2,#3) -- (#2+#4*2/3,#3) -- (#2+#4*2/3,#3+#5/2) -- (#2+#4,#3+#5/2) -- (#2,#3+#5) -- (#2-#4,#3+#5/2) -- (#2-#4*2/3,#3+#5/2) -- (#2-#4*2/3,#3) -- (#2,#3);
    \draw[#1, black] (#2,#3) -- (#2+#4*2/3,#3) -- (#2+#4*2/3,#3+#5/2) -- (#2+#4,#3+#5/2) -- (#2,#3+#5) -- (#2-#4,#3+#5/2) -- (#2-#4*2/3,#3+#5/2) -- (#2-#4*2/3,#3) -- (#2,#3);
}

% draws one entry of the color legend for the program scheme
% color, text, posX, posY (left, top coordinate)
\newcommand{\colorLegendEntry}[4]{
    \filldraw[#1] (#3,#4) rectangle (#3+1,#4+1);
    \draw[black] (#3,#4) rectangle (#3+1,#4+1);
    \node at (#3+0.5,#4+0.4) (color_legend_entry_point) {};
    \node[right=2mm of color_legend_entry_point] (color_legend_entry_text) {#2};
}

% draws a color legend for the program scheme
% posX, posY (left, top coordinate)
\newcommand{\colorLegend}[2] {
    \colorLegendEntry{black!30!orange}{Static translator parts}{#1}{#2}
    \colorLegendEntry{blue}{Intermediate representation (IR)}{#1}{#2-1.5}
    \colorLegendEntry{purple}{Machine code / ELF file}{#1}{#2-3}
}

% draws the schematic presentation of the program
% scale, fontSize
\newcommand{\ProgramSchemeVersionOne}[2]{
    \begin{tikzpicture}[scale=#1]
        % draw color legend
        \colorLegend{-1}{-5}

        % riscv elf file rectangle
        % background
        \filldraw[purple] (-2,-2) rectangle (2,2);
        % black border
        \draw[black] (-2,-2) rectangle (2,2);
        % label
        \node[align=center, color=white] at (0,0) (riscv_text) {\fontsize{#2}{#2} \selectfont RISC-V};

        % lifter arrow
        \TikZArrowX{black!30!orange}{2}{0}{6}{1.5}
        % arrow label
        \node[align=center, color=white] at (4.5,0) (lifter_text) {\fontsize{#2}{#2} \selectfont Lifter};

        % ir (unoptimized) rectangle
        % background
        \filldraw[blue] (8,-2) rectangle (12,2);
        % black border
        \draw[black] (8,-2) rectangle (12,2);
        % label
        \node[align=center, color=white] at (10,0) (ir_text_1) {\fontsize{#2}{#2} \selectfont IR};

        % optimizer arrow
        \TikZArrowX{black!30!orange}{12}{0}{6}{1.5}
        % arrow label
        \node[align=center, color=white] at (14.5,0) (optimizer_text) {\fontsize{#2}{#2} \selectfont Optimizer};

        % ir (optimized) rectangle
        % background
        \filldraw[blue] (18,-2) rectangle (22,2);
        % black border
        \draw[black] (18,-2) rectangle (22, 2);
        % label
        \node[align=center, color=white] at (20,0) (ir_text_2) {\fontsize{#2}{#2} \selectfont IR};

        % compiler arrow
        \TikZArrowX{black!30!orange}{22}{0}{6}{1.5}
        % arrow label
        \node[align=center, color=white] at (24.5,0) (generator_text) {\fontsize{#2}{#2} \selectfont Generator};

        % x86_64 rectangle
        %background
        \filldraw[purple] (28,-2) rectangle (32,2);
        % black border
        \draw[black] (28,-2) rectangle (32,2);
        % label
        \node[align=center, color=white] at (30,0) (x86_64_text) {\fontsize{#2}{#2} \selectfont x86\_64};
    \end{tikzpicture}
}


%%%%%%%%%%%%%%%%%%%%%%%%%%%%%%%%%
%% TODO: Ersetzen Sie in den folgenden Zeilen die entsprechenden -Texte- % mit den richtigen Werten.
\newcommand{\theGroup}{IhreGruppennummer} % Beispiel: 42
\newcommand{\theNumber}{IhreProjektnummer} % Beispiel: A123
% Authors, sorted by last name
\author{Lukas Döllerer \and Jonathan Hettwer \and Johannes Maier \and Tobias Schwarz \and Felix Solcher}
\date{Summer semester 2021}
%%%%%%%%%%%%%%%%%%%%%%%%%%%%%%%%%

% Diese Zeile bitte -nicht- aendern. \title{Gruppe \theGroup{} -- Abgabe zu Aufgabe \theNumber}
\title{Static binary translation from RISC-V to x86\_64}

\begin{document}
\maketitle

\tableofcontents

\pagebreak

\section{Motivation and problem statement}

% Bullet points:
% - why to translate binaries?
%   - to use a program on another platform (e.g. source code lost / not available)
%   - (academic purposes :-))
% - How to translate binaries?
%   - DBT (dynamic binary translation): translate sequence of instructions and execute them
%   - SBT (static binary translation): translate binary to target ISA, then execute it (pro: higher
%     execution time)
% - comparable to difference between interpreted and compiled programs
% - our problem: SBT from RISC-V to x86-64
% - more academic problem, because most programs are available on x86-64

In order to execute machine code, you need a processor which runs this machine code. But there are
many different processor families and instruction set architectures (ISA) for which programs are
developed. If you want to run a program which is written for an ISA where you don't have access to a
system running this ISA, e.g. RISC-V, then you can translate it and execute it on another system,
e.g.\ x86\_64. There are three main approaches:

\par

Firstly you can use an emulator which ``[\ldots] interprets program instructions at
runtime.''\cite{binary_translation} The development is quite easy, but it is a rather slow method.

\par

Secondly, you can use dynamic binary translation (DBT) which works like an emulator but uses caching
to store already translated instruction sequences for future use. This increases the execution speed
compared to an emulator because often used instructions don't need to be translated
twice.\cite{binary_translation}

\par

And thirdly, you can translate the full binary to an ISA which the target system is able to execute.
That way you can separate translation and execution, which means you only have to translate the
binary once for running it as often you need. This increases execution speed compared to DBT. This
approach is called static binary translation (SBT).\cite{binary_translation}

\par

%The difference between SBT and DBT is comparable to the difference between compiled and interpreted
%programs. But whilst compilers and interpreters are working on high level programming languages,
%e.g. C, C++ or Python, SBT and DBT are used to translate machine language files.

\par

Our task is to develop a static binary translator which translates RISC-V binaries to x86\_64
machine code. This field of binary translation is a rather academic issue because currently, x86\_64
is still a more widely used ISA than RISC-V.\cite{riscv_rises} This means most programs are already
available for the x86\_64 processor architecture and therefore there is a small set of set of
programs which need to be translated.

\section{Background}
\subsection{Short overview of RISC-V}

According to the RISC-V specification~\cite{rvspec}, RISC-V is an open and freely accessible ISA
which consists of ``[\ldots] a small base integer ISA [\ldots]'' and ``[\ldots] optional standard
extensions to support general purpose software development''\cite[p.~1]{rvspec}. There is also the
option to implement custom extensions. It is available in ``[b]oth 32-bit and 64-bit address space
variants for applications, operating system kernels, and hardware
implementations.''\cite[p.~1]{rvspec} As the name suggests, the RISC-V ISA describes a \emph{Reduced
    Instruction Set Computer}. This kind of microprocessor is characterized by a larger number of
registers, its load/store architecture, fixed-length instruction words and a generally limited
amount of instructions and instruction formats.\cite{RISCvCISC}

\par

Currently there are standard extensions for multiplication and division (M), for atomic instructions
(A), for floating point arithmetic (F,D,Q), for compressed instructions to reduce the code size (C)
and for control and status registers (Ziscr). More standard extensions are planned.\cite{rvspec}

\subsection{Differences between RISC-V and x86\_64}
In contrast to RISC-V, the x86\_64 ISA describes a CISC, a \emph{Complex Instruction Set Computer}.
Its processors have a limited number of registers, richer instruction sets and variable length
instructions. Because of the complex instruction functions and formats, logic is often implemented
in mix between microcode and hard coded logic, executing instructions over a number of CPU clock
cycles.\cite{RISCvCISC}

\par

The x86\_64 ISA defines instructions which access memory and perform arithmetic or logical
operations at the same time. This is not possible with the load/store RISC-V architecture. Because
of their variable instruction lengths, x86\_64 instructions can contain immediate values which are
up to 64-bit wide.\cite[Vol.~2B~p.~4-35]{intel2017man} RISC-V base instructions are currently always
32-bit wide (16-bit for compressed instructions)\cite[p.~8]{rvspec} which makes loading big
immediates difficult. These are either combined by two or more separate immediate loading operations
or loaded relative to the current instruction pointer.\cite[p.~19]{rvspec} The latter is also often
used for performing jumps.\cite [p.~20]{rvspec}

\par

For the translation this differences are necessary knowledge to be able to generate efficient and
optimized code. For example some instructions need to be assembled or can be merged to a single
instruction in x86\_64.

\subsection{Inspiration}

\section{Approach}\label{approach_section}

We divided the translation process in three main parts: lifting, optimization and generation. We
made this division to structure the program and to simplify the implementation of optimizations.
%% --- im not 100% sure that this is right, please change it if its wrong --- %%
The parts are independent to at least theoretically save the opportunity to support other ISA as
source or target. But this isn't the focus of this project.
%% -------------------------------------- END ------------------------------- %%

\par

The lifter reads the instructions from the binary file and decodes the opcodes using
frvdec\cite{frvdec}. The resulting instructions are used to disassemble the RISC-V binary into basic
blocks, operations and variables in the intermediate representation (IR). It contains the logic of
the input binary in an architecture-neutral form which generic optimization passes can be applied
on. They reduce the number of operations required without altering the program's behavior. The
generator compiles the IR program to x86\_64 assembly. To assemble an executable binary we use the
GNU assembler (AS)\cite{gnu_binutils}. We link our helper library with it using the GNU linker
(LD)\cite{gnu_binutils}. For further explanation, please refer to the following chapters.

\par

Figure \ref{program_overview} depicts a schematic of the basic translator structure. It shows the
operating parts of the translator and the used and created data objects, represented the arrows and
squares respectively.

\begin{figure}
    \centering
    \ProgramSchemeVersionOne{0.43}{13}
    \caption{
        A schematic representation of the translator's internal structure.
    }
    \label{program_overview}
\end{figure}

\par

Our implementation only supports 64-bit RISC-V ELF\footnote{Executable and Linking Format} binaries
which are statically linked and in little endian format. Another requirement is that the binary
conforms to the System V ABI\footnote{Application Binary Interface}. This defines, inter alia, the
system call IDs and the calling convention. These prerequisites are checked, using the metadata
contained in the ELF-file, before translating the file in order to print meaningful debug messages
and to stop the translation.

\section{Intermediate representation}

As explained in \nameref{approach_section}, we use an IR to create a platform-independent
representation of the translated program's logic. This is not only useful to support different input
and output languages in the future, but also to able to optimize the more abstract representation.
By parsing control flow and program structures like basic blocks, we can run optimizations like
\nameref{dead_code_elimination}, \nameref{constant_folding}.

\par

The IR is structured into functions which contain basic blocks. Basic blocks are blocks of code
which don't contain a control flow changing operation (CfOp). CfOps are jumps, subroutine calls,
return statements and system calls. They are used to connect basic blocks with each other. Each
basic block consists of a vector of variables. Each variable is an SSA (\ref{ssa}) variable which is
only assigned a value once by an operation.

\subsection{Single static assignment form}\label{ssa}

First described in 1988 for optimizing an intermediate program representation\cite{ssa_proposal},
the \emph{single static assignment form} (SSA) is used for creating and optimizing explicit control
flow graphs. It follows one basic rule: ``[E]ach variable is 
assigned to exactly once in the program text.''\cite[p.~18]{ssa_proposal} In our IR, each variable
stores its source operation (if it is the result of an operation). This enables us to efficiently
backtrace variable origins and e.g.\ replace variables which can be evaluated statically with constant
immediates (\ref{constant_folding}) or eliminate any unused variables (\ref{dead_code_elimination}).


\section{Lifter}
\subsection{General method}

% Bullet Points:
% - naming: lifter -> elevator: lift to higher code level
% - runs sequentially through the text section
% - storing variables assigned to registers in so called mapping (+ memory token)
% - for each instruction an instruction sequence in the ir is created
% - when discovering a cfop:
%   - finish basic block (adding cfop)
%   - address known:
%       - already scanned address: split basic block or set jump target
%       - not scanned: set as start of a basic block
%   - if unknown:
%       - use backtracking to find possible addresses (controllable through flag)
%       - else target = dummy, let runtime handle
% - last parts of lifting:
%   - add entry block with setup stack
%   - fixup jump targets / predecessors / successors / binary relative immediates

% executeable sections müssen nicht nur instruktionen enthalten, auch daten möglich, von Neumann Daten & Instruktionen selber Speicher

The lifter creates the IR based on the RISC-V binary. This process is called lifting because the
machine code is lifted up to a more abstract representation at a higher level, the IR.

\par

The lifter runs sequentially through all sections of the binary which are marked with the flags
\emph{alloc}, \emph{exec} and \emph{progbits}. This means that they are loaded at runtime into
memory, executable and initialized at load. If there are no sections in the ELF file, the parsed
instructions are gained from the \emph{LOAD} program header which is marked as \emph{read-only} and
\emph{executable} (R+E). This sections respectively program header is loaded because there are
usually the instruction stored.

\par

But due to the von Neumann memory model which specifies that instructions and data
are stored in the same memory, it isn't possible to distinguish instructions and data. For example,
if some constants are defined in the \emph{.text} section which can be interpreted as valid
instructions, they are lifted as instructions. This is also known to be a hard problem in
binary translation.

\par

While lifting the instructions, the so called \emph{mapping} keeps track of the variables which
represents the content of the registers. This is necessary to know to which variables are the inputs
of a newly created operation in the IR. At the start of a basic block the mapping is initialized
with variables from static except from the index zero because this represents the hard wired zero in
RISC-V. Therefore each read from the mapping at index zero is handled by adding an immediate
variable with value zero. Each write doesn't affect the mapping as specified for the \emph{x0}
register.\cite{rvspec}

\par

For each discovered instruction a sequence of variables connected through operations is added to the
IR. These variables represents the functionality of the instruction as defined by the RISC-V manual.

\subsection{Floating point support}
\subsection{Backtracking}\label{backtracking}
Although indirect jump addresses can be evaluated at runtime, we can only jump to the beginning of
detected basic blocks. This is the case, because we can optimize (\ref{dead_code_elimination}) and fold
instructions (\ref{constant_folding}) inside basic blocks without altering the control flow or
logic of the program and as such there might often not be a one-to-one correspondence of RISC-V to x86\_64-Instructions.
Having a completely runnable program thus means detecting every basic block
with their corresponding entry points (including their exit points). Although determining all possible
indirect jump targets statically is not possible (e.g. virtual function calls)
% or not feasible (e.g. jump targets could be influenced by user input)
, backtracking register contents during lifting can provide us with
an approximation of the target set.

\par

To statically determine indirect jump target addresses, we need to evaluate possible contents of
the jump's input register operand. Each variable stores its origin, being either the result of an
operation, an immediate value or from a static. This enables us to backtrack the value of
any variable in the call tree until we reached a variable which can't be backtracked or until we evaluated
all possible values.

\par

Registers keep their values between jumps and calls. We model this behavior using static mappers
which bind variables to their registers in between basic blocks. For backtracking, this means a
variable which is bound to a static mapper can have any value that the variable mapped to the same
static mapper in its basic block's predecessors has. This opens up a room of variables which
influence the jump target, growing with the amount of predecessors and operation inputs.

\par

Figure~\ref{backtracking_graph} shows an example for such a call tree. The basic block \emph{bb24} ends
with an indirect jump, jumping to the address stored in \emph{v37}. \emph{ADD} and \emph{SHL} are operations.

\begin{figure}
    \centering
    \begin{tikzpicture}
        \begin{pgfonlayer}{nodelayer}
            \node [style=BB] (0) at (-3.75, 0) {bb24};
            \node [style=VAR] (1) at (-2.25, 0) {v37};
            \node [style=OP] (2) at (-0.75, 0) {ADD};
            \node [style=VAR] (3) at (0.5, 1.5) {v36};
            \node [style=VAR] (4) at (0.5, -1.5) {v34};
            \node [style=OP] (5) at (2, -1.5) {SHL};
            \node [style=VAR] (6) at (3.5, 0.25) {v15};
            \node [style=VAR] (7) at (3.5, -1.5) {v22};
            \node [style=STATIC] (8) at (5, 0.25) {x14};
            \node [style=STATIC] (9) at (5, -1.5) {x21};
            \node [style=BB] (10) at (6.5, 1) {bb42};
            \node [style=BB] (11) at (6.5, -1) {bb23};
            \node [style=BB] (12) at (6.5, 2) {bb8};
            \node [style=BB] (13) at (6.5, 0) {bb17};
            \node [style=BB] (14) at (6.5, -2.25) {bb127};
            \node [style=VALUE] (15) at (2, 1.5) {-2};
            \node [style=VAR] (16) at (7.75, 2) {};
            \node [style=VAR] (18) at (7.75, 0) {};
            \node [style=VAR] (19) at (7.75, -1) {};
            \node [style=BB] (23) at (9.75, 1) {};
            \node [style=BB] (24) at (9.75, -2.25) {};
            \node [style=OP] (25) at (8.75, 0) {};
            \node [style=OP] (26) at (8.75, -1) {};
            \node [style=OP] (28) at (8.75, 2) {};
            \node [style=VAR] (29) at (7.75, 1) {};
            \node [style=VAR] (31) at (7.75, -2.25) {};
            \node [style=STATIC] (32) at (8.75, 1) {};
            \node [style=STATIC] (33) at (8.75, -2.25) {};
            \node [style=none] (34) at (9.75, 2) {};
            \node [style=none] (35) at (9.75, 1.5) {};
            \node [style=none] (36) at (9.75, 0.25) {};
            \node [style=none] (37) at (9.75, -0.25) {};
            \node [style=none] (38) at (9.75, -0.75) {};
            \node [style=none] (39) at (9.75, -1.25) {};
            \node [style=none] (40) at (9.75, 2.5) {};
        \end{pgfonlayer}
        \begin{pgfonlayer}{edgelayer}
            \draw [style=LINK] (0) to (1);
            \draw [style=LINK] (1) to (2);
            \draw [style=LINK] (2) to (3);
            \draw [style=LINK] (2) to (4);
            \draw [style=LINK] (3) to (15.center);
            \draw [style=LINK] (4) to (5);
            \draw [style=LINK] (5) to (7);
            \draw [style=LINK] (5) to (6);
            \draw [style=LINK] (6) to (8);
            \draw [style=LINK] (7) to (9);
            \draw [style=LINK] (9) to (14);
            \draw [style=LINK] (9) to (11);
            \draw [style=LINK] (8) to (13);
            \draw [style=LINK] (8) to (10);
            \draw [style=LINK] (8) to (12);
            \draw [style=LINK] (12) to (16);
            \draw [style=LINK] (13) to (18);
            \draw [style=LINK] (11) to (19);
            \draw [style=LINK] (16) to (28);
            \draw [style=LINK] (10) to (29);
            \draw [style=LINK] (29) to (32);
            \draw [style=LINK] (18) to (25);
            \draw [style=LINK] (14) to (31);
            \draw [style=LINK] (31) to (33);
            \draw [style=LINK] (33) to (24);
            \draw [style=LINK] (19) to (26);
            \draw [style=LINK] (32) to (23);
            \draw [style=DASHED] (28) to (34.center);
            \draw [style=DASHED] (28) to (35.center);
            \draw [style=DASHED] (25) to (36.center);
            \draw [style=DASHED] (25) to (37.center);
            \draw [style=DASHED] (26) to (38.center);
            \draw [style=DASHED] (26) to (39.center);
            \draw [style=DASHED] (28) to (40.center);
        \end{pgfonlayer}
    \end{tikzpicture}
    \caption{
        Exemplary variable backtracking graph.\\
        \textbf{bb}: BasicBlock; \textbf{v}: Variable; \textbf{x}: StaticMapper;
    }\label{backtracking_graph}
\end{figure}

% TODO: divide this into two parts, one lifter one generator? since there are two distinct parts to it
\subsection{Call and return optimization}
The RISC-V architecture doesn't include special operations for handling subroutines. A function call
is assembled as a jump which places the address of the next instruction into a \emph{link-}register
\emph{x1} or \emph{x5} (for the standard calling convention).\cite[p.~20]{rvspec} Returning from a
subroutine is done by jumping to the address which is contained in either of those registers. This
behavior is further optimized with the compressed instructions \emph{C.JAL} and
\emph{C.JALR}. These operations implicitly store the address of the next instruction
to the link register \emph{x1}.[p.~105]\cite{rvspec}

\par

Subroutine calls which jump to an address in a register are so called \emph{icalls}. They act
similar to indirect jumps and are backtracked in the same way. However, target detection success for indirect calls
doesn't affect stack alignment. Even if we can't determine an indirect call target, we still perform
an x86\_64 \emph{call} which pushes the instruction pointer onto the stack and jumps to the address
in the supplied register. This works as long as we were able to identify the jump address as a basic
block's entry point with the procedure described in the section\ \ref{backtracking}.

\subsection{Jump table detection}

As described in\ \cite{jump_table_paper}, n-conditional case statements, like the C programming
language's switch-case statement, are often implemented by compilers using jump
tables. A jump table is a pattern which enables efficient\footnote{Because the integer condition is
    used as the jump address selector, this is only efficient for switch-statements with small entry
    ranges (e.g.\ enum switching).} branching based
on a stored set of addresses which are possible jump targets. This means we need to detect jump tables to find
all possible basic block entry points.

\par

The jump table pattern always involves consists of a particular set of RISC-V instructions.
At the beginning, a bounds check is executed. A branching instruction is used to either jump to the
default branch or the code after the switch if the switch condition (an integer) is above the
highest case value.

\par

If this is not the case, the switch condition is multiplied by a factor of 4.
Using a combination of a \emph{lui} and an \emph{addi} instruction, it is possible to load a full
32-bit constant\ \cite[p.~19]{rvspec} representing the start address of the jump table.
This address is added to the integer condition. The resulting value is the memory address
at which the desired jump address is stored. A 32-bit load is executed to load the value into a
register. An indirect jump to this register's content is the final instruction, executing the case
label jump.

\par

For the jump table detection, we just backtrack the previously described pattern for every
encountered indirect jump. If the jump turns out to be part of a jump table configuration, we can
extract the jump target addresses and use them as entry points for new basic blocks. This also works
for switch-statements which are compiled into a combination of jump tables and branch-based search
trees.

\section{Optimizations}
\subsection{Dead Code Elimination}\label{dead_code_elimination}

There are cases in compiled programs, where a value is written to a register, but never read from again.
When lifted, these register writes are found in the IR as unused variables, which can be removed.
This is accomplished by visiting the variables in reverse order, and dropping them when their reference-count is zero.
This also takes care of ``cascading'' deletes, as the reference-count for inputs of an operation is decreased once the
operation is deleted.

Another purpose of the Dead Code Elimination pass is to remove unused inputs from BasicBlocks. For this, the
reference-count does not suffice, since there might be circular control flow (like in loops). Instead, the Dead Code
Elimination pass marks each variable associated with side effects, i.e. variables used in \texttt{store}s,
\texttt{syscall}s and jumps to unknown targets like \texttt{ijump}s. This marking is then propagated through the IR,
until no more variables are reachable. Then, all unmarked variables and inputs can be deleted.

\subsection{Constant Folding and Constant Propagation}\label{constant_folding}

Constant Folding and Constant Propagation are closely related optimizations, where operations with known input values
(i.e. immediate values) are computed, and immediate variables are propagated throughout BasicBlocks.
In our implementation, Constant Folding, Constant Propagation and operation simplifications are grouped into the same
pass.  As such, we'll further refer to these collectively as Constant Folding.

This optimization pass is done by visiting each operation in a BasicBlock in order, making the pass run in linear time.
Depending on the inputs of an operation, one of several actions can be performed.

In the simplest case, all inputs are immediates. The result of the operation can be computed, and the variable is
replaced with the resulting immediate value. One exception to this are binary-relative immediates---immediates which are
offset at runtime by the base address of the binary. Hence, only addition and subtraction with one binary-relative and
one non-binary-relative immediate operand are evaluable, and evaluating subtraction is only sensible if the
binary-relative immediate is the first operand.

The next case is one immediate input and one non-immediate (static or operation) input. Generally, this case allows for
simplification of identity operations, like addition with zero. If the non-immediate operand is an operation with an
immediate operand itself, the two immediates can be evaluated under the rules of associativity and commutativity
(Exceptions for binary-relative immediates still apply).

A simple example for the simplification is the \texttt{NOP}-pseudoinstruction in RISC-V, which is usually encoded as
\texttt{ADDI x0, x0, 0}.\cite[p.~20]{rvspec} The addition with zero does not change the value, and all occurrences of
the result variable can be replaced with the input variable (In practice, this simple case can already be detected in
the lifting stage).

This optimization is greatly simplified by the use of Static Single Assignment form in the IR, since variables can't be
reassigned. This means that the value of a variable is known just by looking at the declaration of the variable.

\clearpage

\section{Generator}
The generator has two jobs. % TODO: jobs isnt the right word...
Firstly, it needs to assemble all basic blocks which means assembling all variables with their associated operations,
preparing the inputs for the next basic block and creating the control flow operations.
Secondly, it needs to provide for a runtime environment, namely the original binary in its memory-layouted
form, a stack for the translated binary to use as the x86\_64-Stack is reserved for the generator, a resolver for ijumps
along with an interpreter which acts as a fallback if an indirect jump to an address at which no basic block starts is encountered
and architecture-specific functions like a routine to create an initial stack for the translated binary as well as a routine to translate syscalls.
Most of these are bundled into a static library that is linked to the generated assembly called the Helper-Library.
\subsection{Naive implementation}
For assembly generation, the generator operates in two distinct modes which we refer to as the "naive" implementation and register allocation.
\par
The naive implementation is meant to serve as a simple codegen which is necessary to ease debugging and to better evaluate gains from
optimizations in the IR as it very closely recreates the IR operations.

\par

It generates assembly for each basic block independently.
For each one it creates a stack frame which holds space for every variable in the basic block, iterates all operations and
retrieves its sources - either from a static or from the stack frame -, applies the operation and then saves the results in the stack frame again.
At the end of a block it iterates all control flow operations, evaluates its condition if necessary, writes out the inputs for the next basic block
from the stack frame and either calls a helper routine or directly jumps to the next block.

% TODO: example
\subsection{Register allocation}
\subsection{Merging operations}
\subsection{Interpreter}
\subsection{Helper library}\label{helper}

\section{Performance}
\subsection{Our solution}
\subsection{Optimierungen (Analyse)}
\subsection{QEMU}
\subsection{Dynamic translator last year}
\subsection{Native x86\_64}

\section{Summary}

% TODO: Fuegen Sie Ihre Quellen der Datei Ausarbeitung.bib hinzu Referenzieren Sie diese dann mit
% \cite{}. Beispiel: CR2 ist ein Register der x86-Architektur~\cite{intel2017man}.
\bibliographystyle{plain}
\bibliography{Ausarbeitung}{}

\end{document}
