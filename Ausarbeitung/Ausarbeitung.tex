% Diese Zeile bitte -nicht- aendern.
\documentclass[course=eragp]{aspdoc}

%%%%%%%%%%%%%%%%%%%%%%%%%%%%%%%%%
%% TODO: Ersetzen Sie in den folgenden Zeilen die entsprechenden -Texte-
%% mit den richtigen Werten.
\newcommand{\theGroup}{IhreGruppennummer} % Beispiel: 42
\newcommand{\theNumber}{IhreProjektnummer} % Beispiel: A123
% Authors, sorted by last name
\author{Lukas Döllerer \and Jonathan Hettwer \and Johannes Maier \and Tobias Schwarz \and Felix Solcher}
\date{Sommersemester 2021}
%%%%%%%%%%%%%%%%%%%%%%%%%%%%%%%%%

% Diese Zeile bitte -nicht- aendern.
% \title{Gruppe \theGroup{} -- Abgabe zu Aufgabe \theNumber}
\title{Statische Binärübersetztung von RISC-V nach x86\_64}

\begin{document}
\maketitle

% \section{Einleitung}


% \section{Lösungsansatz}


% TODO: Je nach Aufgabenstellung einen der Begriffe wählen
% \section{Korrektheit/Genauigkeit}


% \section{Performanzanalyse}


% \section{Zusammenfassung und Ausblick}

\section{Motivation / Problemstellung}
\subsection{Unterschiede zwischen RISC-V und x86\_64}

\section{Lösungsansatz}

\section{Intermediate Representation}
\subsection{Motivation für Aufbau}
\subsection{Aufbau \& Struktur \& Inspiration}
\subsection{Constant Folding}
\subsection{Dead Code Elimination}
\subsection{Weitere Optimierungen (falls implementiert)}

\section{Lifter}
\subsection{Allgemeiner Vorgehensweise}
\subsection{Backtracing}
\subsection{Floating Point Unterstützung}
\subsection{Jump Tables}
\subsection{Call and return (falls fertig)}


\section{Generator}
\subsection{Naiver Ansatz}
\subsection{Helper-Library}
\subsection{Register-Allocation}
\subsection{Interpreter}
\subsection{Signals (falls fertig)}
\subsection{Atomics (falls fertig)}

\clearpage

\section{Performance}
\subsection{Translator}
\subsection{Optimierungen (Analyse)}
\subsection{QEMU}
\subsection{Dynamic Translator letztes Jahr}
\subsection{Native RISC-V}
\subsection{Native x86\_64}

\section{Korrektheit?}
\subsection{Optimisierungsfehler}
\subsection{Floating Point Ungenauigkeit}

\section{Fazit}

\section{Glossar}

% TODO: Fuegen Sie Ihre Quellen der Datei Ausarbeitung.bib hinzu
% Referenzieren Sie diese dann mit \cite{}.
% Beispiel: CR2 ist ein Register der x86-Architektur~\cite{intel2017man}.
\bibliographystyle{plain}
\bibliography{Ausarbeitung}{}

\end{document}
