% Diese Zeile bitte -nicht- aendern.
\documentclass[course=eragp]{aspdoc}

%%%%%%%%%%%%%%%%%%%%%%%%%%%%%%%%%
%% TODO: Ersetzen Sie in den folgenden Zeilen die entsprechenden -Texte-
%% mit den richtigen Werten.
\newcommand{\theGroup}{IhreGruppennummer} % Beispiel: 42
\newcommand{\theNumber}{IhreProjektnummer} % Beispiel: A123
% Authors, sorted by last name
\author{Lukas Döllerer \and Jonathan Hettwer \and Johannes Maier \and Tobias Schwarz \and Felix Solcher}
\date{Summer semester 2021}
%%%%%%%%%%%%%%%%%%%%%%%%%%%%%%%%%

% Diese Zeile bitte -nicht- aendern.
% \title{Gruppe \theGroup{} -- Abgabe zu Aufgabe \theNumber}
\title{Static binary translation from RISC-V to x86\_64}

\begin{document}
\maketitle

\section{Motivation and problem statement}

\section{Background}
\subsection{Short overview of RISC-V}
\subsection{Differences between RISC-V and x86\_64}
\subsection{Inspiration}

\section{Approach}

\section{Intermediate representation}

\section{Lifter}
\subsection{General method}
\subsection{Floating point support}
\subsection{Backtracking}
\subsection{Call and return optimization}
\subsection{Jump table detection}

\section{Optimizations}
\subsection{Constant Folding}
\subsection{Dead Code Elimination}

\clearpage

\section{Generator}
\subsection{Interpreter}
\subsection{Naive implementation}
\subsection{Register allocation}
\subsection{Helper library}

\section{Performance}
\subsection{Our solution}
\subsection{Optimierungen (Analyse)}
\subsection{QEMU}
\subsection{Dynamic translator last year}
\subsection{Native x86\_64}

\section{Fazit}

% TODO: Fuegen Sie Ihre Quellen der Datei Ausarbeitung.bib hinzu
% Referenzieren Sie diese dann mit \cite{}.
% Beispiel: CR2 ist ein Register der x86-Architektur~\cite{intel2017man}.
\bibliographystyle{plain}
\bibliography{Ausarbeitung}{}

\end{document}
