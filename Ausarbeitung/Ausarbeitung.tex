% Diese Zeile bitte -nicht- aendern.
\documentclass[course=eragp]{aspdoc}

%%%%%%%%%%%%%%%%%%%%%%%%%%%%%%%%%
%% TODO: Ersetzen Sie in den folgenden Zeilen die entsprechenden -Texte-
%% mit den richtigen Werten.
\newcommand{\theGroup}{IhreGruppennummer} % Beispiel: 42
\newcommand{\theNumber}{IhreProjektnummer} % Beispiel: A123
% Authors, sorted by last name
\author{Lukas Döllerer \and Jonathan Hettwer \and Johannes Maier \and Tobias Schwarz \and Felix Solcher}
\date{Summer semester 2021}
%%%%%%%%%%%%%%%%%%%%%%%%%%%%%%%%%

% Diese Zeile bitte -nicht- aendern.
% \title{Gruppe \theGroup{} -- Abgabe zu Aufgabe \theNumber}
\title{Static binary translation from RISC-V to x86\_64}

\begin{document}
\maketitle

\tableofcontents

\section{Motivation and problem statement}

% Bullet points:
% - why to translate binaries?
%   - to use a program on another platform (e.g. source code lost / not available)
%   - (academic purposes :-))
% - How to translate binaries?
%   - DBT (dynamic binary translation): translate sequence of instructions and execute them
%   - SBT (static binary translation): translate binary to target ISA, then execute it (pro: higher execution time)
% - comparable to diffrence between interpreted and compiled programs
% - our problem: SBT from RISC-V to x86-64
% - more academic problem, because most programs are available on x86-64 

In order to execute machine code, you need a processor which runs this machine code.
But there are many different processor families and instruction set architectures (ISA)
for which programs are developed. If you want to run a program which is desigend for
some ISA, e.g. RISC-V, but you don't have access to a RISC-V system. Then you can
translate it and exectue it on another system, e.g. x86\_64. There are two main approaches
for binary translation: \newline
Fistly you can use dynamic binary translation (DBT) which means to translate short
instruction sequences and then execute them. When discovering a jump or another
control flow operation, the address is evaluated and the translation continues at this
address. But as you do the translation at runtime, this slows down the execution. \newline
Secondly you can translate the whole binary to a binary which the target system can
execute. So you can separate translation and execution. So you only have to translate
the binary once and then can run it as often you needed without translating it each time.
This decreases the execution time. This approach is called static binary translation (SBT).\newline
The difference between SBT and DBT can be compared to the difference between compiled
and interpreted programs. But whilst compiler and interpreter are working on
high level programming languages, e.g. C, C++ or python, SBT and DBT are used to translate
binary files.\newline
Our task is to develop a static binary translator from RISC-V to x86\_64. This is more an
academic issue because x86\_64 is a much more popular plattform than RISC-V and
therefore most programs are available on x86\_64.

\section{Background}
\subsection{Short overview of RISC-V}
\subsection{Differences between RISC-V and x86\_64}
\subsection{Inspiration}

\section{Approach}

\section{Intermediate representation}

\section{Lifter}
\subsection{General method}
\subsection{Floating point support}
\subsection{Backtracking}
\subsection{Call and return optimization}
\subsection{Jump table detection}

\section{Optimizations}
\subsection{Constant Folding}
\subsection{Dead Code Elimination}

\clearpage

\section{Generator}
\subsection{Interpreter}
\subsection{Naive implementation}
\subsection{Register allocation}
\subsection{Helper library}

\section{Performance}
\subsection{Our solution}
\subsection{Optimierungen (Analyse)}
\subsection{QEMU}
\subsection{Dynamic translator last year}
\subsection{Native x86\_64}

\section{Fazit}

% TODO: Fuegen Sie Ihre Quellen der Datei Ausarbeitung.bib hinzu
% Referenzieren Sie diese dann mit \cite{}.
% Beispiel: CR2 ist ein Register der x86-Architektur~\cite{intel2017man}.
\bibliographystyle{plain}
\bibliography{Ausarbeitung}{}

\end{document}
