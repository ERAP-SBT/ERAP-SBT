% Diese Zeile bitte -nicht- aendern.
\documentclass[course=eragp]{aspdoc}

% draws an arrow from (#2,#3) to (#2+#4,#3) with height of #5 (arrow in x direction)
% color (and other arguments, like visible on), startX, startY, length (X dir), height (Y dir)
\newcommand{\TikZArrowX}[5]{
    \filldraw[#1] (#2,#3) -- (#2,#3+#5*2/3) -- (#2+#4/2,#3+#5*2/3) -- (#2+#4/2,#3+#5) -- (#2+#4,#3) -- (#2+#4/2,#3-#5) -- (#2+#4/2,#3-#5*2/3) -- (#2,#3-#5*2/3) -- (#2,#3);
    \draw[#1, black] (#2,#3) -- (#2,#3+#5*2/3) -- (#2+#4/2,#3+#5*2/3) -- (#2+#4/2,#3+#5) -- (#2+#4,#3) -- (#2+#4/2,#3-#5) -- (#2+#4/2,#3-#5*2/3) -- (#2,#3-#5*2/3) -- (#2,#3);
}

% draws an arrow from #2,#3) to (#2,#3+#5) with height of #4 (arrow in y direction)
% color (and other arguments, like visible on), startX, startY, length (X dir), height (Y dir), color
\newcommand{\TikZArrowY}[5]{
    \filldraw[#1] (#2,#3) -- (#2+#4*2/3,#3) -- (#2+#4*2/3,#3+#5/2) -- (#2+#4,#3+#5/2) -- (#2,#3+#5) -- (#2-#4,#3+#5/2) -- (#2-#4*2/3,#3+#5/2) -- (#2-#4*2/3,#3) -- (#2,#3);
    \draw[#1, black] (#2,#3) -- (#2+#4*2/3,#3) -- (#2+#4*2/3,#3+#5/2) -- (#2+#4,#3+#5/2) -- (#2,#3+#5) -- (#2-#4,#3+#5/2) -- (#2-#4*2/3,#3+#5/2) -- (#2-#4*2/3,#3) -- (#2,#3);
}

% draws one entry of the color legend for the program scheme
% color, text, posX, posY (left, top coordinate)
\newcommand{\colorLegendEntry}[4]{
    \filldraw[#1] (#3,#4) rectangle (#3+1,#4+1);
    \draw[black] (#3,#4) rectangle (#3+1,#4+1);
    \node at (#3+0.5,#4+0.4) (color_legend_entry_point) {};
    \node[right=2mm of color_legend_entry_point] (color_legend_entry_text) {#2};
}

% draws a color legend for the program scheme
% posX, posY (left, top coordinate)
\newcommand{\colorLegend}[2] {
    \colorLegendEntry{black!30!orange}{Static translator parts}{#1}{#2}
    \colorLegendEntry{blue}{Intermediate representation (IR)}{#1}{#2-1.5}
    \colorLegendEntry{purple}{Machine code / ELF file}{#1+20}{#2}
}

% draws the schematic presentation of the program
% scale, fontSize
\newcommand{\ProgramScheme}[2]{
    \begin{tikzpicture}[scale=#1]
        % draw color legend
        \colorLegend{-1}{-6}

        % translator rectangle
        \draw[black, thick] (2.5,4) rectangle (27.5, -4);
        \node[align=center, color=black] at (15, 3) (translator_text) {\fontsize{#2}{#2} \selectfont Translator};

        % riscv elf file rectangle
        \draw[black, fill=purple] (-2,-2) rectangle (2,2);
        % label
        \node[align=center, color=white] at (0,0) (riscv_text) {\fontsize{#2}{#2} \selectfont RISC-V \\ \fontsize{#2}{#2} \selectfont Binary};

        % lifter arrow
        \TikZArrowX{black!30!orange}{2}{0}{6}{1.5}
        % arrow label
        \node[align=center, color=white] at (4.5,0) (lifter_text) {\fontsize{#2}{#2} \selectfont Lifter};

        % ir (unoptimized) rectangle
        \draw[black, fill=blue] (8,-2) rectangle (12,2);
        % label
        \node[align=center, color=white] at (10,0) (ir_text_1) {\fontsize{#2}{#2} \selectfont IR};

        % optimizer arrow
        \TikZArrowX{black!30!orange}{12}{0}{6}{1.5}
        % arrow label
        \node[align=center, color=white] at (14.5,0) (optimizer_text) {\fontsize{#2}{#2} \selectfont Optimizer};

        % ir (optimized) rectangle
        \draw[black, fill=blue] (18,-2) rectangle (22, 2);
        % label
        \node[align=center, color=white] at (20,0) (ir_text_2) {\fontsize{#2}{#2} \selectfont IR};

        % compiler arrow
        \TikZArrowX{black!30!orange}{22}{0}{6}{1.5}
        % arrow label
        \node[align=center, color=white] at (24.5,0) (generator_text) {\fontsize{#2}{#2} \selectfont Generator};

        % x86_64 rectangle
        \draw[black, fill=purple] (28.5,-2) rectangle (32.5,2);
        % label
        \node[align=center, color=white] at (30.5,0) (x86_64_text) {\fontsize{#2}{#2} \selectfont Trans- \\ \fontsize{#2}{#2} \selectfont lated \\ \fontsize{#2}{#2} \selectfont Binary};

        % helper rectangle
        \draw[black, fill=purple] (28.5, 9) rectangle (32.5, 5);
        % helper label
        \node[align=center, color=white] at (30.5, 7) (helper_text) {\fontsize{#2}{#2} \selectfont Helper \\ \fontsize{#2}{#2} \selectfont Library};

        % connection between x86_64 and helper library
        \draw[black, -{Stealth[scale=3]}] (30, 2) -- (30, 5);
        \draw[black, {Stealth[scale=3]}-] (31, 2) -- (31, 5);

        % result binary rectangle
        \draw[black, thick] (28, 12) rectangle (33, -4);
        % result binary label
        \node[align=center, color=black] at (30.5, 10.5) (result_binary_text) {\fontsize{#2}{#2} \selectfont Result \\ \fontsize{#2}{#2} \selectfont Binary};
        
        % memory image arrow + label
        \draw[black, -{Stealth[scale=3]}] (0, 2) -- (0, 7) -- (28, 7);
        \node[align=center, color=black] at (15, 8) {\fontsize{#2}{#2} \selectfont Memory Image};
    \end{tikzpicture}
}

\input{listing_styles/riscv_lstlisting_style.tex}

\usepackage{placeins}

%%%%%%%%%%%%%%%%%%%%%%%%%%%%%%%%%
%% TODO: Ersetzen Sie in den folgenden Zeilen die entsprechenden -Texte- % mit den richtigen Werten.
\newcommand{\theGroup}{IhreGruppennummer} % Beispiel: 42
\newcommand{\theNumber}{IhreProjektnummer} % Beispiel: A123
% Authors, sorted by last name
\author{Lukas Döllerer \and Jonathan Hettwer \and Johannes Maier \and Tobias Schwarz \and Felix Solcher}
\date{Summer semester 2021}
%%%%%%%%%%%%%%%%%%%%%%%%%%%%%%%%%

% Diese Zeile bitte -nicht- aendern. \title{Gruppe \theGroup{} -- Abgabe zu Aufgabe \theNumber}
\title{Static binary translation from RISC-V to x86\_64}

\begin{document}
\maketitle

\tableofcontents

\pagebreak

\section{Motivation and Problem Statement}

% Bullet points:
% - why to translate binaries?
%   - to use a program on another platform (e.g. source code lost / not available)
%   - (academic purposes :-))
% - How to translate binaries?
%   - DBT (dynamic binary translation): translate sequence of instructions and execute them
%   - SBT (static binary translation): translate binary to target ISA, then execute it (pro: higher
%     execution time)
% - comparable to difference between interpreted and compiled programs
% - our problem: SBT from RISC-V to x86-64
% - more academic problem, because most programs are available on x86-64

% TODO: Ahead Of Time

In order to natively execute machine code, a processor which runs this machine code is needed. But
there are many different processor families with different instruction set architectures (ISA) for
which programs are developed. To run a program which is written for an ISA one does not have access
to a system running this ISA, one can translate it and execute it on another system. There are three
main approaches to do this:

\par

The first option for executing programs written for a different ISA is to use an emulator. It
emulates a different execution environment by interpreting the program's instructions
just-in-time~\cite{binary_translation}.
The development is simple, but it is a rather slow method.

\par

The second option is to use dynamic binary translation (DBT). It works like an emulator but uses caching
to store already translated instruction sequences for future use. This increases the execution speed
compared to an emulator because often used instructions do not need to be translated
multiple times.~\cite{binary_translation}

\par

As the third option, one can statically translate the full binary. The program is rewritten for the
target system's ISA.
Translation and execution are separated phases, which means the binary only has to be translated once for
running it multiple times. This increases execution speed compared to DBT. This
approach is called static binary translation (SBT).~\cite{binary_translation}

\par

The current rise of RISC-V processors makes it probable that we will see a need for RISC-V to
x86\_64 binary translation in the near future. Even though x86\_64 is still a more widely used ISA
than RISC-V, a lot of companies invest in a shift towards the RISC-V ISA. This revolution makes ISA
incompatibilities a future problem we want to solve today.~\cite{riscv_rises} Another important use
case for the SBT is translating binaries where the source code was lost. Programs without source
code can not be compiled for another target ISA, but they can be translated using SBT.

\par

Therefore we developed a static binary translator from RISC-V to x86\_64 which uses an intermediate
representation (IR). The instruction decoding and code generation are separated by using as the IR
as a abstract data structure. Our aim is to be faster than the popular dynamic binary translator
QEMU.

\par

The following chapters will provide some background information for the RISC-V architecture, the
relevant differences
to the x86\_64 architecture (Section~\ref{sec:background}) and our general
approach to static binary translation (Section~\ref{sec:approach}). Afterwards, the distinct parts of the translator, namely our
intermediate representation (Section~\ref{sec:ir}), the RISC-V lifter
(Section~\ref{sec:riscv_lifter}), the optimizer (Section~\ref{sec:optimizations}), and the x86\_64
code generator (Section~\ref{sec:generator}) are further explained.
Section~\ref{sec:floating_point} describes our approach to support floating point RISC-V
instructions and the present difficulties. The last section (Section~\ref{sec:evaluation})
evaluates the quality and performance of our implementation using benchmarks tests.

\section{Background}\label{sec:background}

This chapter provides important background information, which is required for our approach. This
includes an overview over the RISC-V instructions, standard calling convention~\footnote{When
    referring to the \textit{standard calling convention}, the RVG calling convention is meant.} and registers,
as well as relevant differences between the RISC-V and the x86\_64 architecture.

\subsection{Short Overview of RISC-V}

According to the RISC-V specification, RISC-V is an open and freely accessible ISA,
which consists of a base ISA for integer computations. It also offers general purpose software development
supporting extensions, which are optional. There is also the
option to implement custom extensions. The RISC-V ISA is available for building 32-bit and 64-bit based
applications and operating system kernels. Because of its minimalistic nature, it is well suited for
custom hardware implementations.~\cite{rvspec}

\par

Currently, there are standard extensions for multiplication and division (M), for atomic
instructions (A), for floating point arithmetic (F,D), for compressed instructions to reduce the
code size (C) and for control and status registers (Ziscr). Together with the RV64 Base
ISA, they are called \texttt{rv64imacfd} or short \texttt{rv64g}. More standard extensions are planned.~\cite{rvspec}

\par

RV64I provides 31 64-bit wide integer registers (\texttt{x1} $-$ \texttt{x31}) and a hard wired zero register (\texttt{x0}). The floating
point extensions add 32 floating points registers (\texttt{f0} $-$ \texttt{f31}) and a floating point control and status
register (\texttt{fcsr}). The instruction words have a fixed width of 32-bit, with the C extension
introducing instructions with 16-bit wide instruction words. The
standard calling convention defines \texttt{x1} as the link register and \texttt{x5} as the alternative link
register. Link registers store the return address of jumps which target a subroutine and intend to
return the parent routine.~\cite{rvspec}

\subsection{Differences between RISC-V and x86\_64}

When comparing the RISC-V and the x86\_64 ISAs, most of the differences can be derived from their
different design ideologies. The x86\_64 ISA is built to describe a CISC, a \emph{Complex Instruction Set Computer}.
This decision results in processors that have a limited number of registers, richer instruction sets and variable length
instructions. Because of the complex instruction functions and formats, logic is often implemented
in a mix between microcode and hard coded logic, executing instructions over a number of CPU clock
cycles.~\cite{RISCvCISC}

\par

As the name suggests, the
RISC-V ISA describes a \emph{Reduced Instruction Set Computer}. This kind of microprocessor is
characterized by a larger number of
registers, its load/store architecture, fixed-length instruction words and a generally limited
amount of instructions and instruction formats.~\cite{RISCvCISC} 

\par

These major design differences play an important role during the binary translation. The following
paragraph describes two relevant differences in greater detail.

\par 

The RISC-V ISA defines a load/store architecture, meaning that a single instruction can either
manipulate the memory or perform an operation that does not access the memory. The x86\_64 ISA
does not have this restriction and defines arithmetic and logical instructions which operate
directly on memory. Also, RISC-V base instructions are currently always
32-bit wide (16-bit for compressed instructions) which makes loading wide
immediates difficult. These are either combined by two or more separate immediate loading operations
or loaded relative to the current instruction pointer.~\cite{rvspec} In comparison, x86\_64 instructions
can contain immediate values which are up to 64-bit wide, because
of their variable instruction lengths~\cite{intel2017man}.

\par

Assemblers make this difference less complicated for developers by introducing so called
\textit{pseudoinstructions}. These are instructions which perform simple tasks like loading an
immediate, jumping to an address, or calling a subroutine. Most of them can't be directly expressed
as a single RISC-V instruction and are thus replaced with a common pattern. This gives a translator
the power to detect these patterns and efficiently convert them to a more compact, x86\_64 representation.

\par

These differences are necessary knowledge for the translation. They give the opportunity to generate efficient and
optimized code. For example, some RISC-V instruction sequences can be merged to a single x86\_64
instruction.

\subsection{Inspiration for the Intermediate Representation}

Languages for intermediate representations (IR) are often used in compilers, which means that highly
advanced concepts already exist. The LLVM assembly language is a established IR, known for its deep
integration into the LLVM ecosystem with existing compiler and debugger backends and its well
defined language reference. These advantages however have a price. The LLVM assembly language is a
complex IR with a lot of features which are not required for the purposes of this project. The used IR
is inspired by the LLVM assembly language, tailored to serve the use case of lifting binaries. The
parallels between our IR and the LLVM assembly language are the static
typing, single static assignment (SSA) variables and machine-instruction-like operations.

\section{Approach}\label{sec:approach}

The translation process is divided into three main parts: instruction lifting, optimization, and code
generation. This partition is made to structure the translator and to simplify the implementation of
optimizations. This allows supporting other source or target ISAs in the future.
Only some modular components need to be extended or replaced for extending the ISA support of this translator.

\par

The IR is tailored to support efficient binary translation. It is the data structure which
contains the logic of the input binary in an architecture-neutral form. Generic optimization
passes can be applied on this structure. They reduce the number or complexity of operations without altering the
program's behavior.

\par

The lifter reads the instruction bytes from the binary file and decodes them using
\texttt{frvdec}~\cite{frvdec}. The resulting instruction objects are used to disassemble the RISC-V binary
into basic blocks, operations, and variables in the IR. The code generator creates x86\_64
gnu assembler assembly, based on the optimized program in the form of the IR.

\par

A helper library is used to implement platform specific or runtime based
functionality which is used by the translated binary. It includes, among other functions, an interpreter which
can be used to translate instructions during runtime. The helper library is designed for specific
source and target ISAs. System calls and the interpreter work in an architecture-specific way.

\par

To create the translated, executable binary, the GNU assembler (AS)~\cite{gnu_binutils} and the GNU
linker (LD)~\cite{gnu_binutils} are used to assemble the generated assembly and link it to the helper library.

\par

Figure~\ref{fig:program_overview} depicts a schematic of the basic structure of the translator. It shows the
operating parts of the translator and the processed data fragments, represented by arrows and
squares respectively.

\begin{figure}[H]
    \centering
    \ProgramScheme{0.43}{13}
    \caption{A schematic representation of the translator's internal structure as well as in- and
        output binaries.}\label{fig:program_overview}
\end{figure}

\par

Our implementation only supports 64-bit RISC-V ELF\footnote{Executable and Linking Format} binaries
which are statically linked and in little endian format. Another requirement is that the binary
conforms to the System V ABI\footnote{Application Binary Interface}. This defines, among other things, the
system call IDs and the calling convention. These prerequisites are checked, using the metadata
contained in the ELF-file, before translating the file. This is required debugging and assertion purposes.

\section{Intermediate Representation}\label{sec:ir}

As explained in Section~\ref{sec:approach}, an IR is used to create a platform-independent
representation of the translated program's logic. This is not only useful to support different input
and output languages in the future, but also to be able to optimize the more abstract representation.
By parsing control flow and program structures, optimizations like
\nameref{dead_code_elimination} and \nameref{constant_folding} can be applied to the IR code.

\begin{figure}[H]
    \begin{center}
        \begin{tabular}{|c | c | c|}
            \hline
            Integers & Floats & Special \\ [0.5ex]
            \hline\hline
            i64      & f64    & mt      \\
            \hline
            i32      & f32    &         \\
            \hline
            i16      &        &         \\
            \hline
            i8       &        &         \\
            \hline
        \end{tabular}
        \caption{Types defined for the IR variables. The numbers describe the width of each type in bits. The
            memory token (\textit{mt}) is a purely virtual type and thus doesn't have a specified width.}\label{ir:types:figure}
    \end{center}
\end{figure}


The IR consists of variables which are statically typed. The types are shown in
Figure~\ref{ir:types:figure}. The variables follow a single static assignment form, which makes them
immutable after they are initialized. Operations take variables as inputs and assign values to
new output variables. Every operation performs one of the available instructions. These are abstract,
assembly-like instructions defining mostly arithmetic and logical operations. A full list of all the
supported instructions can be found in the Figure~\ref{figure:ir_instructions}.
Variables and their operations are grouped into basic blocks. These are blocks of code
which do not contain a control flow changing operation. These special operations are jumps and
system calls. Subroutine calls which are defined by many ISAs are also jump instructions with
additional memory accessing operations. The control flow changing operations act as connectors
between basic blocks. Figure~\ref{figure:cfops} shows a list of all defined control flow changing operations.

\begin{figure}
    \begin{center}
        \def\arraystretch{1.5}
        \begin{tabular}{r | p{0.8\linewidth}}
            \hline
            IR Operation & Description                                                                   \\ [0.5ex]
            \hline
            jump         & Direct jump to a known target address.                                        \\
            ijump        & Indirect jump to a possibly unknown target address.                           \\
            cjump        & Conditional jump to a known target address.                                   \\
            call         & Subroutine call. Also encodes a continuation address to which the subroutine
            is expected to return to.                                                                    \\
            icall        & Same behavior as a normal \textit{call}, but with unknown subroutine target
            address.                                                                                     \\
            return       & Exit subroutine execution and continue at the continuation address of the
            \textit{call} instruction which lead to the subroutine execution.                            \\
            unreachable  & This operation should never be reached. On reach, it terminates the program's
            execution.                                                                                   \\
            syscall      & A syscall in respect to the System V ABI.                                     \\
            \hline
        \end{tabular}
        \caption{Control flow changing operations defined by the IR.}\label{figure:cfops}
    \end{center}
\end{figure}

\par

Each basic block stores a list of its predecessors and successors. Predecessors are the basic blocks which jump to
the current basic block, and successors are the basic blocks that are jumped to by the current basic
block. This ``jump'' can be any control flow changing operation, including syscalls.

\par

A basic block also stores a list of input parameters. These basic block parameters are used for
transferring data between basic blocks. Every variable which is linked to a static mapper (\ref{statics}) is
added to the basic block parameters. These have to be filled with the correct values with every jump to
a new basic block.


\subsection{Single Static Assignment Form}\label{ssa}

First described in 1988 for optimizing an intermediate program representation\cite{ssa_proposal},
the \emph{single static assignment form} (SSA) is used for creating and optimizing explicit control
flow graphs. It follows one basic rule: ``[E]ach variable is
assigned to exactly once in the program text.''~\cite{ssa_proposal} In our IR, each variable
stores its source operation (if it is the result of an operation). This enables us to efficiently
backtrack variable origins and e.g.\ replace variables which can be evaluated statically with constant
immediates (Section~\ref{constant_folding}) or eliminate unused variables (Section~\ref{dead_code_elimination}).

We introduce a \emph{memory token}, a required input for operations which read from memory and the
result of operations which write to memory. It creates a dependency between operations that store
and operations that load data. This should be used to keep the memory access operations
in their correct order when reordering operations inside basic blocks and functions. A writing
operation acts as a separator of reading operations. Because of the SSA variables used in the IR,
correct memory access order is automatically achieved as long as no variable is referenced before
its assignment. This insures a consistent memory access ordering and therefore the correctness of the
program.

\subsection{Static Mappers}\label{statics}

To keep track of register contents between basic blocks, we introduced a concept called a
\textit{static mapper} or a \textit{static} for short.
Variables which link to them can be used as inputs in basic blocks, They act as a signal to code
generators that, if the control flow is transferred to the basic block from an unknown predecessor, 
the variable can be expected in this static.
For control flow operations with an unknown target, a mapping from the output of the basic 
block to the static at which they should be stored is then created.

\par

The IR itself does not enforce a fixed set of statics but instead leaves it to the lifter to 
create as many statics as it needs to lift the source ISA.
For RISC-V, these are 31 general purpose registers, one memory token, 32 floating point registers
and one control and status flag register. The register \texttt{x0} is always zero and thus not
assigned to a static mapper.

\section{RISC-V Lifter}\label{sec:riscv_lifter}

This chapter describes how the RISC-V instructions are decoded and parsed, how they are lifted
into the intermediate representation and which optimizations can be made during lifting. This
includes the subroutine call optimization, indirect jump target backtracking and jump table detection.

\subsection{General Method}

% Bullet Points:
% - naming: lifter -> elevator: lift to higher code level
% - runs sequentially through the text section
% - storing variables assigned to registers in so called mapping (+ memory token)
% - for each instruction an instruction sequence in the ir is created
% - when discovering a cfop:
%   - finish basic block (adding cfop)
%   - address known:
%       - already scanned address: split basic block or set jump target
%       - not scanned: set as start of a basic block
%   - if unknown:
%       - use backtracking to find possible addresses (controllable through flag)
%       - else target = dummy, let runtime handle
% - last parts of lifting:
%   - add entry block with setup stack
%   - fixup jump targets / predecessors / successors / binary relative immediates

% executeable sections müssen nicht nur instruktionen enthalten, auch daten möglich, von Neumann Daten & Instruktionen selber Speicher

The lifter creates the IR based on the RISC-V binary. This process is called lifting because the
machine code is lifted up to a more abstract representation at a higher level, the IR.

\subsubsection{Program Decoding}\label{sec:program_decoding}

The translator starts with the ELF input binary file which is supplied by the user. After checking the
validity of the binary file and whether it is suited for being translated by the program, the
RISC-V lifter decodes all instructions.

\par

ELF binaries store information about what data is stored where in the binary using \textit{program
    headers} and \textit{section headers}. Both header types store information about the virtual address
spaces, allocation details and access flags. The ELF-file specification does not guarantee that every ELF
binary defines \textit{sections}~\cite{elf_spec}. The lifter therefore supports instruction
detection using either \textit{program headers} or \textit{section headers}. However,
\textit{section headers} hold finer and more detailed information
about program sections than the \textit{program headers} and are therefore preferred for instruction
decoding. 

\par

The ELF-file loader parses instructions from the sections of the binary
which are marked as executable. In case there are no sections in the ELF file\footnote{e.g.\ memory dumps.}, we use the
program loader information, stored in the \emph{program headers}. They
specify which bytes of the underlying binary are actually loaded to the final program.\footnote{Program header type \texttt{PT\_LOAD}.}
We only parse instructions from the program headers which are marked as \emph{readable} and
\emph{executable}.

\par

Due to the memory architecture of modern computers, instructions and data are mostly stored in the same
memory. This concept, which was first introduced by John von Neumann in 1945\cite{vna}, makes it
hard to statically distinguish between instructions and data. That is why we rely on the additional
information stored in the section- and program headers which identifies certain parts of the binary
as executable instructions. This however means we rely on the compiler and the programmer to use
executable sections only for storing instructions.

\subsubsection{IR Generation}

After loading the data form the binary, the lifter runs sequentially through the discovered
instructions which are decoded using \texttt{frvdec}\cite{frvdec}. The resulting instructions are objects,
filled with the parsed content of the binary instruction code. This is a
more generic and easier processable form with direct access to involved register numbers and
immediate values. For each of the instructions, a variable amount of operations and variables is added to
the IR to replicate the original program's behavior.

\begin{figure}
    \begin{subfigure}{\textwidth}
        \centering
        \begin{lstlisting}[language={[RISC-V]Assembler}]
            start:
                [...]
            loop:
    0xb0:       addi a0, a0, -1
    0xb4:       bne a0, zero, loop
            continuation:
    0xb8:       sll a1, a1, a2
                [...]
        \end{lstlisting}
\caption{RISC-V assembly sequence used as an example for the lifter. Decrements \texttt{a0} by one
in a loop until it is equal to zero,
            then shifts \texttt{a1} to the left by the amount stored in \texttt{a2}.}\label{fig:lifting_example_riscv}
        \vspace{1em}
    \end{subfigure}
    \begin{subfigure}{\textwidth}
        \centering
        \begin{lstlisting}
[...]

block b1(/*input list */) <= [/* predecessors */] { // loop
    [...]
    i64 v9 <- @10
    [...]
    imm v33 <- immediate -1
    i64 v35 <- add i64 v9, imm v33
    imm v36 <- immediate 0
    imm v37 <- immediate 0xb0
    imm v38 <- immediate 0xb8
} => [(cjump, [b1, /* target inputs */], eq, v35, v36, v37), (jump, [b2, /* target inputs */], v38)]

block b2(/* input list */) <= [b1, /* other predecessors */] { // continuation
    [...]
    i64 v10 <- @11
    i64 v11 <- @12
    [...]
    i64 v33 <- shl i64 v10, i64 v11
} => [/* control flow op */]

[...]
    \end{lstlisting}
        \caption{The IR created by the lifter from the RISC-V assembly in Figure~\ref{fig:lifting_example_riscv}. \\
            Some duplicated or unnecessary parts, e.g. predecessors list, variables from statics, target
            inputs and control flow operations are omitted.}\label{fig:lifting_example_ir}
    \end{subfigure}
    \caption{Exemplary lifting of a RISC-V assembly program.}
\end{figure}

\par

Figure \ref{fig:lifting_example_riscv} shows an example RISC-V assembly sequence and Figure
\ref{fig:lifting_example_ir} shows the resulting IR parts.

\par

As explained in the IR chapter (Section~\ref{sec:ir}), our IR is subdivided into basic blocks which contain
consecutive instructions without a change in control flow. As soon as a control flow changing
instruction is discovered, it is associated with the basic block. A basic block can be
associated to one or more closing control flow changing
instructions, e.g. RISC-V branches which consist of a conditional
jump and a direct jump to the next instruction (in the next block) if the branch is not taken. The current basic block is finished and
a new one is created. If the jump address can be inferred from the instruction using
Backtracking (Section~\ref{backtracking}) or if it is directly encoded, like in branch instructions, this address is used
to register the start of a new basic block.

\par

If the instructions at the virtual target address have not been parsed by the lifter
yet\footnote{This is the case for forward jumps.}, the basic block entry address is stored.

Otherwise, if the lifter already parsed the instruction at the target address without starting a new
basic block, the block containing the jump address must be split up at the jump address.
This is required because it is only possible to jump to the start of a basic block.

\par

Splitting basic blocks on backwards jumps is required because of the way the lifter consumes the
input binary. It
runs over all instructions twice. Once for decoding them and a second time for lifting them into the
IR. The second run includes detecting indirect jumps and parsing basic blocks at the same time. This
causes every backwards jump which would have changed the basic block parsing to be detected too late. It then has to
change the already existing basic blocks by splitting them at the jump's target address.

\par

When lifting instructions, the lifter stores their original virtual address. This is used for
splitting basic blocks. During splitting, all instructions with an original address lower than the split address are
sorted into the first block, the other ones into the second. Both blocks are connected using a
direct jump. The predecessor and successor lists as well as some operation inputs need to be
adjusted.

\par

If the jump address cannot be extracted from the instruction, the backtracking (Section~\ref{backtracking})
is used to infer possible addresses which are processed with the same procedure as described before.
Using a command line flag, the user can control whether all possible
addresses should be found or whether only the first one should be used. If no jump address can be inferred, a dummy
basic block which indicates an unresolved jump target, is set as the target block. The jump address then
needs to be evaluated at runtime.
\par

If a jump target address is inside a decoded instruction, the further behavior of the program is undefined.

\par

While lifting the instructions, a register file, called \emph{mapping}, keeps track of the variables
which represents the content of all registers, the memory token, and the control and status registers
(csr). This is required to determine the inputs of the operations based on the input registers of
the instructions. For example, when lifting a RISC-V instruction which has the registers \texttt{x13}
and \texttt{x14} as inputs, the input variables are taken from the mapping at the corresponding
indices.

\par

At the start of a basic block the mapping is initialized
with IR variables from statics (Section~\ref{statics}). The register with the index zero is a hard wired zero in
RISC-V and therefore not initialized with a variable. Furthermore, each read from the mapping at index zero is handled by adding an immediate
variable with value zero. Each write to the zero register is ignored, as it is specified for the \texttt{x0}
register.\cite{rvspec}

\subsubsection{Postprocessing}

After all available instructions have been lifted, a post processing step is required to connect
the basic blocks with each other. Because we lift the file by reading the instructions
consecutively, a basic block does not know its successors during lifting. Vice versa, a new basic
block does not know its full list of predecessors during lifting.
During post processing, we iterate over all basic blocks, assign them their jump target basic
blocks and adjust their predecessor and successor lists.

\par

The last step in lifting is to add an entry basic block that sets the RISC-V stack up and jumps to the
basic block at the ELF-file's entry address. This virtual address is stored in the ELF header
and specifies where the execution of the binary starts.\cite{elf_spec}

\subsection{Atomic Instructions}

To support programs which use atomic instructions from the ``A'' standard extension, the lifter converts
every atomic instruction into a set of non-atomic instructions which emulate the original
instruction's behavior. This does not change the logic or validity of the program as long as the
instructions are executed consecutively, without the need for atomic ressource access.


\subsection{Backtracking}\label{backtracking}
Although indirect jump addresses can be evaluated at runtime, we can only jump to the beginning of
detected basic blocks. This is the case, because we can optimize (Section~\ref{dead_code_elimination}) and fold
instructions (Section~\ref{constant_folding}) inside basic blocks without altering the control flow or
logic of the program. That is why there might often not be a one-to-one correspondence of RISC-V to x86\_64-Instructions.
Creating a fast and runnable program thus means detecting most basic blocks
with their corresponding entry points (including their exit points). Although determining all possible
indirect jump targets statically is not possible (e.g.\ virtual function calls),
% or not feasible (e.g. jump targets could be influenced by user input)
backtracking register contents during lifting can provide us with
an approximation of the indirect jump target set.

\par

To statically determine indirect jump target addresses, we need to evaluate possible contents of
the jump's input register operand. Each variable stores its origin, being either the result of an
operation, an immediate value or from a static. This creates a dependency tree.
To backtrack the value of a variable, this tree has to be traversed until a variable which can not
be backtracked is reached or until all possible values have been evaluated.

\par

Registers keep their values between jumps and calls. We model this behavior using static mappers
which bind variables to their registers in between basic blocks. For backtracking, this means a
variable which is bound to a static mapper can have any value that the variable mapped to the same
static mapper in its basic block's predecessors has. This opens up a room of variables which
influence the jump target, growing with the amount of predecessors and operation inputs.

\par

Figure~\ref{backtracking_graph} shows an example for such a dependency tree.

\begin{figure}
    \centering
    \begin{tikzpicture}
        \begin{pgfonlayer}{nodelayer}
            \node [style=BB] (0) at (-3.75, 0) {bb24};
            \node [style=VAR] (1) at (-2.25, 0) {v37};
            \node [style=OP] (2) at (-0.75, 0) {ADD};
            \node [style=VAR] (3) at (0.5, 1.5) {v36};
            \node [style=VAR] (4) at (0.5, -1.5) {v34};
            \node [style=OP] (5) at (2, -1.5) {SHL};
            \node [style=VAR] (6) at (3.5, 0.25) {v15};
            \node [style=VAR] (7) at (3.5, -1.5) {v22};
            \node [style=STATIC] (8) at (5, 0.25) {x14};
            \node [style=STATIC] (9) at (5, -1.5) {x21};
            \node [style=BB] (10) at (6.5, 1) {bb42};
            \node [style=BB] (11) at (6.5, -1) {bb23};
            \node [style=BB] (12) at (6.5, 2) {bb8};
            \node [style=BB] (13) at (6.5, 0) {bb17};
            \node [style=BB] (14) at (6.5, -2.25) {bb127};
            \node [style=VALUE] (15) at (2, 1.5) {-2};
            \node [style=VAR] (16) at (7.75, 2) {};
            \node [style=VAR] (18) at (7.75, 0) {};
            \node [style=VAR] (19) at (7.75, -1) {};
            \node [style=BB] (23) at (9.75, 1) {};
            \node [style=BB] (24) at (9.75, -2.25) {};
            \node [style=OP] (25) at (8.75, 0) {};
            \node [style=OP] (26) at (8.75, -1) {};
            \node [style=OP] (28) at (8.75, 2) {};
            \node [style=VAR] (29) at (7.75, 1) {};
            \node [style=VAR] (31) at (7.75, -2.25) {};
            \node [style=STATIC] (32) at (8.75, 1) {};
            \node [style=STATIC] (33) at (8.75, -2.25) {};
            \node [style=none] (34) at (9.75, 2) {};
            \node [style=none] (35) at (9.75, 1.5) {};
            \node [style=none] (36) at (9.75, 0.25) {};
            \node [style=none] (37) at (9.75, -0.25) {};
            \node [style=none] (38) at (9.75, -0.75) {};
            \node [style=none] (39) at (9.75, -1.25) {};
            \node [style=none] (40) at (9.75, 2.5) {};
        \end{pgfonlayer}
        \begin{pgfonlayer}{edgelayer}
            \draw [style=LINK] (0) to (1);
            \draw [style=LINK] (1) to (2);
            \draw [style=LINK] (2) to (3);
            \draw [style=LINK] (2) to (4);
            \draw [style=LINK] (3) to (15.center);
            \draw [style=LINK] (4) to (5);
            \draw [style=LINK] (5) to (7);
            \draw [style=LINK] (5) to (6);
            \draw [style=LINK] (6) to (8);
            \draw [style=LINK] (7) to (9);
            \draw [style=LINK] (9) to (14);
            \draw [style=LINK] (9) to (11);
            \draw [style=LINK] (8) to (13);
            \draw [style=LINK] (8) to (10);
            \draw [style=LINK] (8) to (12);
            \draw [style=LINK] (12) to (16);
            \draw [style=LINK] (13) to (18);
            \draw [style=LINK] (11) to (19);
            \draw [style=LINK] (16) to (28);
            \draw [style=LINK] (10) to (29);
            \draw [style=LINK] (29) to (32);
            \draw [style=LINK] (18) to (25);
            \draw [style=LINK] (14) to (31);
            \draw [style=LINK] (31) to (33);
            \draw [style=LINK] (33) to (24);
            \draw [style=LINK] (19) to (26);
            \draw [style=LINK] (32) to (23);
            \draw [style=DASHED] (28) to (34.center);
            \draw [style=DASHED] (28) to (35.center);
            \draw [style=DASHED] (25) to (36.center);
            \draw [style=DASHED] (25) to (37.center);
            \draw [style=DASHED] (26) to (38.center);
            \draw [style=DASHED] (26) to (39.center);
            \draw [style=DASHED] (28) to (40.center);
        \end{pgfonlayer}
    \end{tikzpicture}
    \caption{
        Schematic dependency tree of an indirect jump target address variable (\textbf{v37}). The basic block \textbf{bb24} ends
        with an indirect jump, jumping to the address stored in \texttt{v37}. \texttt{ADD} and \texttt{SHL}
        are operations, names starting with \textbf{x} are names of static mappers.\\
    }\label{backtracking_graph}
\end{figure}

% TODO: divide this into two parts, one lifter one generator? since there are two distinct parts to it

\subsection{Jump Table Detection}

As described in\ \cite{jump_table_paper}, n-conditional case statements, like the C programming
language's switch-case statement, are often implemented by compilers using jump
tables. A jump table is a pattern which enables efficient\footnote{Because the integer condition is
    used as the jump address selector. This is only efficient for switch-statements with small entry
    ranges (e.g.\ enum switching).} branching based
on a stored set of addresses which are possible jump targets. This means we need to detect jump tables to find
all possible basic block entry points.

\par

The jump table patterns observed in generated assembly always consist of a particular set of RISC-V instructions.
At the beginning, a bounds check is executed. A branching instruction is used to either jump to the
default branch or the code after the switch if the switch condition (an integer) is above the
highest case value.

\par

If this is not the case, the switch condition is multiplied by a factor of 4.
Using a combination of a \texttt{LUI} and an \texttt{ADDI} instruction, it is possible to load a full
32-bit constant\cite{rvspec} representing the start address of the jump table.
This address is added to the integer condition. The resulting value is the memory address
at which the desired jump address is stored. A 32-bit load is executed to load the value into a
register. An indirect jump to this register's content is the final instruction, executing the case
label jump.

\par

For the jump table detection, the previously described pattern is backtracked for every
encountered indirect jump. If the jump turns out to be part of a jump table configuration, the jump
target addresses can be extracted and used as entry points for new basic blocks. This also works
for switch-statements which are compiled into a combination of jump tables and branch-based search
trees.

\subsection{Call and Return Optimization}
The RISC-V architecture does not include special operations for handling subroutines. A function call
is assembled as a jump which places the address of the next instruction into a \emph{link-}register
\texttt{x1} or \texttt{x5} (for the standard calling convention).\cite{rvspec} Returning from a
subroutine is done by jumping to the address which is contained in either of those registers. This
behavior is further optimized with the compressed instructions \texttt{C.JAL} and
\texttt{C.JALR}. These operations implicitly store the address of the next instruction
to the link register \texttt{x1}.\cite{rvspec}

\par

Subroutine calls which jump to a dynamic address are so called \emph{icalls}. They act
similar to indirect jumps and are backtracked in the same way.

\section{Optimizations}\label{sec:optimizations}

The IR produced by the lifter is not always optimal. For example, at the time of lifting, the lifter is not always able
to know whether a variable will be used elsewhere, or whether some operation can be directly calculated or combined with
another operation that follows later. For these purposes, separate optimization passes are performed once the full IR is
available.

\subsection{Dead Code Elimination}\label{dead_code_elimination}

There are cases in compiled programs, where a value is written to a register, but never read again.
When lifted, these register writes are found in the IR as unused variables, which can be removed through \emph{Dead Code
    Elimination} (DCE). For this purpose, each variable has a reference count assigned to it. This is a number, which is
increased each time the variable is referenced in another operation or control flow operation. If the reference count of
a variable is zero, it is not used anywhere and can be removed.

By visiting the variables in reverse order, \textit{cascading} deletes are taken care of, as the reference count for
inputs of an operation is decreased once the operation is deleted.

Another purpose of the Dead Code Elimination pass is to remove unused inputs from basic blocks. For this, the
reference-count does not suffice, since there might be circular control flow (like in loops). Instead, DCE marks each
variable associated with side effects, i.e.\ variables used in \texttt{store}s, \texttt{syscall}s and jumps to unknown
targets like \texttt{ijump}s. This marking is then propagated through the IR by visiting operation inputs and
predecessors, until no more variables are reachable. Then, all unmarked variables and inputs can be deleted. This
technique can be compared to a graph search algorithm, which detects connected components containing an operation with
side effects.

\begin{figure}[H]
    \centering
    \begin{lstlisting}[]
block b1(i64 v0, i64 v1) <= [/* predecessors */] {
    i64 v0 <- @0 (1)
    i64 v1 <- @1 (3)
    imm v2 <- immediate -1 (3)
    i64 v3 <- add i64 v0, imm v2 (1)
    imm v4 <- immediate 0 (2)
    i64 v5 <- add i64 v1, imm v4 (0)
} => [(cjump, [b2, v2, v1], eq, v3, v4), (jump, [b1, v2, v1])]

block b2(i64 v0, i64 v1) <= [b1] {
    /* content omitted for simplicity */
    i64 v0 <- @0 (1)
    i64 v1 <- @1 (0)
} => [/* control flow operations */];
\end{lstlisting}
    \caption{Example IR before DCE: The value in parenthesis after each variable is its reference count.}\label{fig:dce_example_before}
\end{figure}

\begin{figure}[H]
    \centering
    \begin{lstlisting}[]
block b1(i64 v0) <= [/* predecessors */] {
    i64 v0 <- @0 (1)
    imm v1 <- immediate -1 (1)
    i64 v2 <- add i64 v0, imm v1 (3)
    imm v3 <- immediate 0 (1)
} => [(cjump, [b2, v2], eq, v2, v3), (jump, [b1, v2])]
block b2(i64 v0) <= [b1] {
    /* content omitted for simplicity */
    i64 v0 <- @0 (1)
} => [/* control flow operations */];
\end{lstlisting}
    \caption{Example IR after DCE.}\label{fig:dce_example_after}
\end{figure}

Figure~\ref{fig:dce_example_before} shows an example IR which has not been optimized. Figure~\ref{fig:dce_example_after}
shows a potential result of performing DCE on this example. To better visualize the operation of DCE, the variables
have been annotated with their corresponding reference count.

The simplest change is variable \texttt{v5}, which is not referenced in any other variable or control flow operation,
and can be eliminated. One can also see that \texttt{v1} is not used, but since it is still passed
as a target input to
other basic blocks, its reference count is not zero. In this case, the marking process finds that \texttt{v1} is not
involved in any side effect, and consequently removes it. Since it is also a static, the corresponding input is removed
from the basic block.

\subsection{Constant Folding and Constant Propagation}\label{constant_folding}

Constant Folding and Constant Propagation are closely related optimizations, where operations with known input values
(i.e.\ immediate values) are computed, and immediate variables are propagated throughout basic blocks.
In our implementation, Constant Folding, Constant Propagation, and operation simplifications are grouped into the same
pass. As such, they are further referred to collectively as Constant Folding.

This optimization pass is done by visiting each operation in a basic block in order, making the pass run in linear time.
If the variable is an integer arithmetic, logic or conditional set operation, one of several actions can be performed.

In the simplest case, all inputs are immediates. The result of the operation can be computed, and the variable is
replaced with the resulting immediate value. One exception to this are binary-relative immediates
--- immediates which are
offset at runtime by the base address of the binary. Hence, only addition and subtraction with one binary-relative and
one non-binary-relative immediate operand are evaluable, and evaluating subtraction is only sensible if the
binary-relative immediate is the first operand. Another exception are instructions with side effects
like store and load instructions, since these still
need to be performed at runtime.

The next case is one immediate input and one non-immediate (static or operation) input. Generally, this case allows for
simplification of identity operations, like addition with zero. If the non-immediate operand is an operation with an
immediate operand itself, the two immediates can be evaluated under the rules of associativity and commutativity
(Exceptions for binary-relative immediates still apply).

This optimization is greatly simplified by the use of the SSA form in the IR, since variables can not be
reassigned. This means that the value of a variable is known just by looking at the declaration of the variable.

\begin{figure}[H]
    \centering
    \begin{lstlisting}[]
block b1(i64 v0) <= [/* predecessors */] {
    i64 v0 <- @0

    imm v1 <- immediate 73
    i64 v2 <- add i64 v0, imm v1
    imm v3 <- immediate 64
    i64 v4 <- add i64 v2, imm v3

    imm v5 <- immediate 95
    imm v6 <- immediate 31
    i64 v7 <- sub imm v5, imm v6

    imm v8 <- immediate 0
    i64 v9 <- add i64 v7, imm v8
} => [/* control flow operations */]
\end{lstlisting}
    \caption{Example IR before Constant Folding.}\label{fig:const_folding_example_before}
\end{figure}

\begin{figure}[H]
    \centering
    \begin{lstlisting}[]
block b1(i64 v0) <= [/* predecessors */] {
    i64 v0 <- @0

    imm v3 <- immediate 137
    i64 v4 <- add i64 v0, imm v3

    i64 v7 <- immediate 64
} => [/* control flow operations */]
\end{lstlisting}
    \caption{Example IR after Constant Folding.}\label{fig:const_folding_example_after}
\end{figure}

Figures~\ref{fig:const_folding_example_before} and \ref{fig:const_folding_example_after} show an example IR before and
after applying Constant Folding. First, variables \texttt{v2} and \texttt{v4} are collapsed into one variable by
evaluating the add with immediates (since addition is associative), making \texttt{v1} and \texttt{v2} dead
code. Next, \texttt{v7} is directly evaluated, since both inputs are immediates. Last, since \texttt{v9} is an add with
zero and thus an identity operation, it is removed, and references to it are replaced with \texttt{v7}.

\subsection{Common Subexpression Elimination}

Common Subexpression Elimination, or more specifically, \textit{(Local) Value
    numbering}~\cite{ssa_proposal}, is an optimization that removes redundant code. In principle, each variable is given a
symbolic value depending on its inputs and operation, variables with the same value are redundant. These redundant
variables can thus be all replaced by the first occurrence. In our implementation, a hash set is employed (similar to
how this optimization is described in the SSA proposal), where relevant properties, i.e. operation type, input
variables, rounding mode, or in the case of immediate variables the value itself, are considered and compared.

Common subexpression elimination is mostly useful for eliminating duplicate immediates, which can also be produced by
the Constant folding pass. Since input binaries are often already optimized, duplicate operations are rather uncommon.

\begin{figure}[H]
    \centering
    \begin{lstlisting}[]
block b1(i64 v0, i64 v1) <= [/* predecessors */] {
    i64 v0 <- @0
    i64 v1 <- @1

    imm v2 <- immediate 64
    i64 v3 <- add i64 v0, imm v2
    imm v4 <- immediate 64
    i64 v5 <- add i64 v0, imm v4

    i64 v6 <- add i64 v1, i64 v5
} => [/* control flow operations */]
\end{lstlisting}
    \caption{Example IR before Common Subexpression Elimination.}\label{fig:cse_example_before}
\end{figure}

\begin{figure}[H]
    \centering
    \begin{lstlisting}[]
block b1(i64 v0, i64 v1) <= [/* predecessors */] {
    i64 v0 <- @0
    i64 v1 <- @1

    imm v2 <- immediate 64
    i64 v3 <- add i64 v0, imm v2

    i64 v6 <- add i64 v1, i64 v3
} => [/* control flow operations */]
\end{lstlisting}
    \caption{Example IR after Common Subexpression Elimination.}\label{fig:cse_example_after}
\end{figure}

\section{Code Generator}\label{sec:generator}
The code generator is used to assemble all basic blocks, which means assembling all variables with their associated operations,
preparing the inputs for the next basic block, and creating the control flow operations for each block.
Additionally, it needs to provide for a runtime environment.
This runtime environment consists of the original binary so that relevant data for execution (e.g. the .data section) is still accessible.
A stack which is used as a replacement for the original stack is also part of the generated runtime environment.
This is required because the x86\_64-Stack is reserved for storing temporary variables when assembling a basic block and return addresses for the Call-Return-Optimization.
% Furthermore, a resolver for indirect jumps is included. An interpreter which acts as a fallback if
% an indirect jump to an address at which no basic block starts is encountered is also available at runtime.
% Architecture-specific functions like a routine to create an initial stack as required by the
% original binariy's system convention as well as a routine to translate syscalls are included.
% The architecture-independent parts are emitted by the generator in the assembly with the rest being part of the Helper Library which is linked against the generated assembly.

\par

When generating assembly for the basic blocks, a user can choose between a simple approach that generates unoptimized assembly
and a more sophisticated approach that generates optimized assembly.

\subsection{Simple Code Generation}\label{naive_generator}

The simple implementation is meant to ease debugging
and to better evaluate gains from
optimizations in the IR as it very closely recreates the IR operations.
As it generates unoptimized assembly, it should only be used for development purposes.

\par

The simple code generator creates assembly for each basic block independently.
It creates stack frames which hold space for every variable in the basic block. It iterates all
basic block operations and
retrieves its sources, either from a static or from the stack frame. The generator applies the
operation and saves the results back in the stack frame.
At the end of a block, it iterates over all control flow operations and evaluates their condition if
necessary. It then writes out the inputs for the next basic block 
from the stack frame and either calls a helper routine or directly jumps to the next block.

\par

To show an example of the code produced by the naive code generator the IR shown in Figure~\ref{fig:example_ir} is used.
This simple IR which decrements a counter until it is 0 produces the assembly shown in Figure~\ref{fig:naive_gen_asm}.
\begin{figure}
    \centering
    \begin{lstlisting}[]
block b1(i64 v0) <= [/* predecessors */] {
    i64 v0 <- @0
    imm v1 <- immediate -1
    i64 v2 <- add i64 v0, imm v1
    imm v3 <- immediate 0
} => [(cjump, [b2, v2], v2, v3, eq), (jump, [b1, v2])]
\end{lstlisting}
    \caption{Example IR used to show the difference between the simple code generator and the register allocator.}\label{fig:example_ir}
\end{figure}

\begin{figure}[h]
    \centering
    \begin{lstlisting}[language={[x86masm]Assembler}]
b1:
    sub rsp, 32        ; create stack frame, 4 variables * 8 bytes for each
    mov rax, [s0]      ; calculate the val of v0
    mov [rsp], rax     ; mov v0 to the stack frame
    mov [rsp + 8], -1  ; immediates are loaded directly into the stack frame
    mov rax, [rsp]     ; operands are loaded from the stack again
    mov rbx, [rsp + 8]
    add rax, rbx       ; v2
    mov [rsp + 16], rax
    mov [rsp + 24], 0  ; v3 is a 0 immediate
    mov rax, [rsp + 16]
    mov rbx, [rsp + 24]
    cmp rax, rbx       ; test cjump condition
    jne b1_cf_1        ; jump to next cfop if not fulfilled
    mov rax, [rsp + 16]
    mov [s0], rax      ; Store input for the next basic block
    add rsp, 32        ; Destroy stack frame
    jmp b2
b1_cf_1:
    mov rax, [rsp + 16]
    mov [s0], rax
    add rsp, 32
    jmp b1
\end{lstlisting}
    \caption{Assembly produced by the simple code generator from Figure~\ref{fig:example_ir}.}\label{fig:naive_gen_asm}
\end{figure}

\subsection{Advanced Code Generation}
To achieve good performance and code density, a more sophisticated approach to generating assembly from the IR needs to be employed.
This is done by using register allocation to keep variables often used in registers and on the stack without spilling them into the statics when transferring control follow
to the next basic block. Additionally we merge certain operation sequences in the IR into single x86\_64-Instructions to achieve better code density and performance.

\subsubsection{Register Allocation}\label{reg_alloc}
Basic blocks which closely represent functions of the original binary are grouped and compiled
together. The register and stack state (so the location of variables) is maintained between blocks.
As such, often used variables are kept in registers to prevent spilling of variables to statics at the end of each basic block.
The achieve this, all blocks which are targets of a call control flow changing operation must first
be marked. The register allocator then iterates over
all basic blocks, checks if they are either the target of a call operation or have no known predecessors.
It then starts allocating registers for the detected group of each basic block.
\par

The allocator compiles the starting block by allocating registers and stack slots for the
variables. It evaluates the control flow changing operations and checks whether the target of such an operation is part of the current group.
This basically means that it is not a target of a call operation. If the target is part of the
current group, it creates an input mapping for the target which specifies where each input variable
is located. However if the block has already been compiled,
the allocator emits a sequence of instructions which move the variables to the position at which the target block expects them.
After this, the handling of the current block is finished and if any of the targets of the block's
control flow changing operations are part of the current group they are recursively compiled too.

\par

The process of generating instructions for the basic block itself is also relatively straight forward. First,
a rough estimate of the time-of-uses of
each variable is created. This is required because control flow changing operations might need
reshuffling.
Afterwards, each operation in the basic block is then iterated over and its inputs are loaded into
registers if they are located in a static or on the stack frame. A destination register is
chosen too.
If possible, a destination register other than the input registers is used if the inputs are still required later. Otherwise, one of the input registers is overwritten.
When choosing a register, empty or disused registers are preferred. In case all registers contain variables which
are still needed,
the variable which is the farthest away from being used again is chosen to be evicted to the stack.
This is evaluated using the stored time of next use.

\par

A possible improvement here is to create a preference for a register for each variable if it is
required at a fixed location. This could be the case if it is used as an input for an already compiled basic block.
This preference could then be taken into account when choosing a new register for storing a variable.

\par

As an example of how this works, we assume that only \texttt{rax} and \texttt{rbx} are available for allocation in the
following example. We also assume that an \texttt{add} of a variable located in \texttt{rax} and an
immediate 10 is compiled.
It is also assumed that the variable in rax is accessed later in the same block again.
If \texttt{rbx} contains no variable which is needed anymore the following code is compiled:
\begin{lstlisting}[language={[x86masm]Assembler},float=h]
lea rbx, [rax + 10] ; rbx is chosen as the destination since it is empty
\end{lstlisting}
\FloatBarrier

If \texttt{rbx} contains a variable and it is used in the operation following the \texttt{add},
\texttt{rax} is chosen as the destination, because it is used after rbx:
\begin{lstlisting}[language={[x86masm]Assembler},float=h]
mov [rsp], rax      ; rax is saved to the stack frame to be retrievable later on
add rax, 10         ; then overwrite it
\end{lstlisting}
\FloatBarrier

If \texttt{rbx} contains a variable but it is only used after the variable contained in
\texttt{rax}, it is instead chosen as the destination:
\begin{lstlisting}[language={[x86masm]Assembler},float=h]
mov [rsp], rbx      ; rbx is saved to the stack frame to be retrievable later on
lea rbx, [rax + 10] ; then overwrite it
\end{lstlisting}

\FloatBarrier
\par

After all blocks in a group are compiled and the final stack-frame size is known the control-flow operations for each block can be assembled.
The difference to the simple generator is that the inputs for the next block might not be written to statics but instead need to be placed in certain register or stack slots
while not overwriting values that are currently stored in them and might be needed later.
This is why the initial time-calculation might be wrong for these cases as it is hard to predict when such
conflicts might occur. The current approach taken is to place the input into statics first, then the stack slots, then the registers while storing values which may be overwritten in temporary stack slots.
Also, when compiling the control flow operations and the target is a block not part of the current group the stack frame size for it might be different and a such it needs to be adjusted as well to prevent stack over- or underflows.

\par

As an example of the input relocating we again assume only rax and rbx to be available which both hold variables used as inputs for the next blocks.
We assume to compile the control flow operations for a block which conditionally jumps to one block with the inputs needed in rbx and the second stack slot respectively.
The other other block that is jumped to assumes the variables in rax and rbx.
\begin{lstlisting}[language={[x86masm]Assembler},float=h]
cmp rax, 0      ; evaluate cjump condition
jne b1_cf_1
; cf 0, adjust rax and rbx to rbx and the second stack slot
mov [rsp + 8 * 2], rbx ; variables needed in the stack frame are written out first
mov rbx, rax           ; now rbx is safe to be overwritten
jmp b2
b1_cf_1:
; no need for adjustment as the locations for the input match
jmp b3
\end{lstlisting}

\FloatBarrier

To show an example of the generated assembly the same IR as in Figure~\ref{fig:example_ir} is used.
The produced assembly will look like in Figure~\ref{fig:reg_gen_asm}.
This is a lot more efficient than the naive generator and leads to vastly increased performance as well as smaller binaries.
\begin{figure}[h]
    \centering
    \begin{lstlisting}[language={[x86masm]Assembler}]
b1:
    mov rax, [s0]   ; Input is in s0 so load it from there
    add rax, -1     ; the immediate operand can be encoded directly into the instruction
    cmp rax, 0      ; evaluate the eq-condition of the cjump control flow operation
    jne b1_cf_1
    mov [s0], rax   ; move the modified var to the location the block expects it at
    jmp b1
b1_cf_1:
    ; b2 expects the input in rax so no need to move it
    jmp b2
\end{lstlisting}
    \caption{Assembly produced by register allocation from Figure \ref{fig:example_ir}.}\label{fig:reg_gen_asm}
\end{figure}

\subsubsection{Translation Blocks}\label{translation_blocks}
One problem that is encountered when employing register allocation on groups of basic blocks is that the programm cannot simply jump to them using indirect jumps anymore
because the basic block might expect variables to be outside of statics and the stack frame, which is set up at the first block of a group, would be missing. To still make it possible for indirect jumps to jump to basic blocks which are register-allocated translation blocks need to be generated.
These translation blocks will set up the stack frame and move the inputs for the basic block to the registers or stack slots at which they are expected before jumping to the register-allocated block.
A problem with these blocks is that they need to be generated for a lot of basic blocks and take up a lot of space in the finished binary (sometimes even more than 30\% of the \.text section) while also being rarely used
since most indirect jumps will end up at function starts which we recognize with relatively good accuracy and blocks after a function call which will be returned to.
So by only generating translation blocks for targets of function returns the binary size can be drastically reduced while also having a negliable impact on performance.

\subsubsection{Merging Operations}
As x86 is a CISC-Architecture it offers a lot more sophisticated instructions than RISC-V or our IR. As such multiple operations in the IR can be merged into a single x86-Instruction
if the intermediate results of them are not needed later.
Mergeable sequences include:
\begin{itemize}
    \item add, (cast), store: Can be merged to a single store with the add embedded in the memory operand and the cast expressed as the store width
    \item add, load, sign-/zero-extend: Can be merged into a single movsx/movzx with the add embedded in the memory operand
    \item and with 0x1f/0x3f,(cast),shr/shl/sar: Can be merged into a single shift instructions since they mask the count operand by themselves
\end{itemize}
Merging these operations not only reduces the binary size since fewer instructions need to be encoded but also increase performance significantly.
% TODO: test the performance diff for merging ops

\subsection{Call and Return Optimization}
To support calls and returns the x86\_64-Stack is used. When the code generator encounters a call it will first destroy the stack frame of the current basic block leaving only the return addresses on the stack,
push the return address from the original binary and a call-Instruction to the target is emitted.
A return is assembled to a compare of the actual return address computed and the previously pushed original return address
along with a ret-Instruction that gets executed when the return addresses match.
When they do not match, the stack pointer is resetted to the original stack pointer before execution of any basic block and the indirect jump handler is invoked.
To prevent underflow 16 bytes of zeroes are pushed onto the stack before the first basic block is executed. This will cause the compare at a return with any address to fail and the ijump handler to be used instead.
To prevent overflow the stack pointer is compared to a maximum allowed size when a call control flow operation is executed and the stack is resetted if it exceeds this.

\subsection{Helper Library}\label{helper}

The ABI that is exposed by Linux at runtime is different on x86\_64 compared to RISC-V. Some aspects are specified by
the System V ABI, like the ELF file format and x86\_64 calling convention, but the system call ABI used by Linux differs
from platform to platform, mostly due to backwards compatibility or architectural differences~\cite{man_syscalls}.

Because of these incompatibilities, some system calls need a thin wrapper, which is defined in the Helper Library. In
most cases, the only difference is the ID used to perform the system call, but in some cases, arguments need to be
modified before the system call can be performed or to be handled specially.
However, system calls related to signals were stubbed to do nothing, since signal support was not necessary for running
the benchmarks and implementing these requires invasive changes.

% TODO: cite something that shows that system calls are slow anyway. => slow helper library does not matter very much
At startup, Linux puts information on the x86\_64 stack, for example the program arguments and auxiliary vectors. % TODO: define what auxiliary vectors are and why it matters ?
Before executing the translated code, these values need to be copied to the emulated Stack, and in some cases transformed.
For example, the auxiliary vectors contain information on where to locate the Linux vDSO~\cite{man_vdso}.
This is an optional ELF object that provides user-space wrappers for some system calls to make them faster (e.g. by not
invoking the rather expensive kernel system call procedure), and is automatically linked by Linux (if present).
The application can use the auxiliary vectors to locate this object and call the provided functions; however,
the ABI is very different, we opted to remove any reference to the vDSO from the auxiliary vectors.
In the future, a vDSO wrapper could be implemented to provide the same speed benefits.

Additionally, the helper library houses more required components for execution, like helpers for indirect jump lookup
(\ref{ijump_resolution}) and the interpreter (\ref{interpreter}).

% bullet points:
% * why necessary (requirements): not everything can be translated ahead of time, sort of done
% * system v abi at startup
%   * ignoring the linux vdso
%   * translate arguments / environment to risc-v stack
% * translate syscalls at startup
%   * often pass through
%   * id mismatch
%   * argument layout mismatch
% * helper functions for ijump hash table lookup -> reference to other chapter
% * contains the interpreter
% * evaluate how well the requirements where reached, what is missing, problems encountered.

\subsection{Indirect Jump Resolution}\label{ijump_resolution}

When encountering an indirect jump, the jump's input variable contains the jump target address in the
original RISC-V binary address space. We try to statically identify most of the indirect jump target
addresses during lifting and are thus able to start new basic blocks at such addresses.
This however means we need to translate the original binary's addresses to basic block start
addresses in the translated binary, at runtime. We also need a way to identify
indirect jump target addresses which are not the start address of any lifted basic block.

We can only jump to the start of a (translation) basic block, because this is the only reference point where the
emulated RISC-V registers are stored at a known location (in the form of statics).

In case we encounter such an unknown jump target address, we switch to the interpreter (Section~\ref{interpreter}) to
interpret the original RISC-V code just-in-time until the next start of a basic block is reached.

\par

Mapping the RISC-V indirect jump target addresses to translated basic blocks requires a
dictionary-like data structure.
The implementation has to have fast access times as we need to access entries with
every indirect jump and every interpreted control flow changing operation in the interpreter. It also has to be able
to identify invalid keys, to switch to the interpreter if the
jump target address is not a valid basic block start address. We developed two data structure and
compared their strengths and weaknesses:

\begin{enumerate}
    \item For the naive implementation, we store a specific 64-bit wide number for each lifted RISC-V address
          in the \texttt{.ro\_data} section. This recreates part of the original binary's address
          space with a known offset. The numbers are either 0 if no basic block with the given start address
          exists or the address of the basic block which starts with instructions from this RISC-V address.
          The latter is done by using the basic block labels, with the assembler calculating the actual
          address numbers.
          \par
          This method has the advantage of having a constant access time with barely no calculation overhead. It
          however means that we need to store additional information in our output assembly file in the size of the
          lifted instructions section of the original binary. This could be part of the problem of the
          assembler AS requiring a lot of memory for assembling the output assembly.

    \item To reduce this size overhead, we implemented a method for creating a nearly perfect hash table with a
          process called ``Hash, displace, and compress''\cite{CHD}. The process consist of a two-stage hash
          function which first sorts the items into buckets and later generates a hash function for each
          bucket. The hashed values are stored in one common hash table. The calculation starts with a load
          factor ($\alpha$) of 100\% and a bucket size of 10 elements ($\lambda$) per bucket in average. These values are lowered
          incrementally by 10\% and 1 element respectively if no valid bucket-level hash function could be found for the
          set constraints.
          While a lower average bucket size decreases the translation time (because hash functions hash
          fewer items which generally results in fewer collisions), a higher average bucket size decreases the translated binary's size (because
          every bucket needs to store its own hash function).
          The bucket-level hash functions only need a 16-bit index to be reconstructed and can
          thus
          be stored memory efficiently. These bucket hash function indices are stored in a separate table in
          the \texttt{.ro\_data} section, together with the hash table.

          \par

          For the hash function generation, we did not use truly random hash functions. The authors of the ``Hash,
          displace, and compress'' perfect hashing method suggest using a ``heuristic hash
          function''\cite{CHD}, one of Bob Jenkins' first published hash functions.\cite{jenkins_hash_1} For
          our implementation, we use one of his latest hash functions which is supposed to be faster than his
          first proposals.\cite{jenkins_hash_2} Because we are only interested in hashing 64-bit addresses, we
          only need part of his hash function which generates hashes for byte arrays of variable sizes.

          \par

          The hashing algorithm is used in the generator for building the hash table and finding the required
          hash functions. It is also implemented in the helper library (Section~\ref{helper}) for providing hashes for addresses
          which are only available during runtime.

          \par

          With the generated hash table and hash functions, every key is assigned an index in the hash table,
          no matter if it is a valid key which was used to build the hash table or not. That is why we store
          both the key and the value, which is the target basic block label in this case, in the hash table.
          For key validation, we first load the key from the calculated index and compare it to the input key.
          If they do not match, we know that no basic block exists for the supplied RISC-V address. Otherwise,
          we load the value from the table which is placed just 64-bits after the key.

          \par

          The hashing helps reduce the required storage space to the hash table size and the hash function
          indices table size. (Figure~\ref{hashing_figure}) The storage size optimizations mean however that
          every access takes, although its complexity is still constant, the additional time of calculating
          the hash table index.
\end{enumerate}

\subsection{Interpreter}\label{interpreter}

An emulator is the slowest method of translating a binary~\cite{binary_translation}. Our interpreter
is an non optimized emulator.\footnote{Our emulator is called \textit{interpreter} in the implementation
    and therefore we use this naming here also.} So the interpreter is comparatively slow as it decodes the
instructions at runtime using frvdec~\cite{frvdec}. In order to access the original RISC-V
instructions at runtime, the original binary has to be stored as data in the result binary.

\par

We did not optimize the interpreter much because it is only a fall back system for the translated
binary and not the focus of this project. Additionally the simple implementation reduces the codes
size of the interpreter and therefore the translated binary size.

\par

The interpreter is called if an indirect jump or indirect call target was not resolved and therefore
the generator does not know where to jump. So the interpreter is used to fill this gap and to jump
back to the translated instructions when possible. This can be done when the start of an basic block
is reached.

\par

When the interpreter is entered and left, the RISC-V register values (represented by the static
values) need to be in the statics. This is important as the generator stores the variable values in
registers or in the stack frame, but the interpreter works on the statics. This needs to be done as
the interpreter needs to know an location where the variables values are stored. So the values are
transferred to the statics before calling the interpreter.

\par

This also an reason why we only leave the interpreter at the begin of an basic block because there
the naive generator (Section~\ref{naive_generator}) takes the basic blocks inputs there from the statics.
When using the register allocator (Section~\ref{reg_alloc}) the translation blocks
(Section~\ref{translation_blocks}) move the values to the correct registers and stack slots and are
therefore --- as well as the top level basic blocks --- the addresses where we can leave the
interpreter. These addresses are stored in the ijump lookup table (Section~\ref{ijump_resolution}).

\par

Usually a translated basic block will be found after few instructions, resulting in almost
negligible slow down from entering the interpreter. However in some extreme cases where no return
basic block is found, practically the entire program is emulated resulting in extreme slow downs.
This can be observed in rare cases if no translation blocks are created resulting in less possible
return targets.

\par

The interpreter fully emulates \texttt{rv64imacfd} and therefore supports the same standard
extensions as the translator. If a command flag is supplied to the translator no
basic blocks will be translated and all code will be interpreted. This was used to verify the
correctness of the interpreter using the benchmark tests. But as explained this method is rather
slow and therefore is not desirable to use as translate binaries.

\par

The interpreter theoretically supports the execution of RISC-V JIT compilers. Using the interpreter
alone, self-modifying code should also be able to be executed. But such programs were not tested and
therefore this is only a theoretical feature.

\par

At runtime the CPUID~\cite{intel2017man} is checked for support of the FMA3 CPU extension. If it is
present the interpreter uses it for faster floating point enulation, otherwise the fused multiply
add instructions are assembled using separate multiplication and addition respectively subtraction
instructions.

\par

\section{Floating Point Handling and Problems}\label{sec:floating_point}

Floating point arithmetic is special as it is affected by rounding and therefore depend on the
implementation of the architecture. We decided to use the x86\_64 SSE instructions if possible and
else assemble them because a bitwise correct implementation is slower. Due to rounding effects the
result may not be equal to the RISC-V implementation.

\subsection{Lifting and Generating}

% TODO: Move footnote content into normal text?
To support the \texttt{F} and \texttt{D}\footnote{The Q standard extension is not supported due to simplicity and lack of usage.}
standard extensions, the mapping as well as the amount of statics have to be extended to include the
floating point registers and the floating point control and status register (fcsr). To lift the
RISC-V floating point instructions, some operations are reused, e.g. \emph{add}, \emph{sub}, and some floating
point specific are defined, e.g. \emph{fmul}, \emph{fsqrt}. The floating point variables have the
types \emph{f32} and \emph{f64} which represents single and double precision numbers, respectively.

\par

The code generator (Section~\ref{sec:generator}) uses the x86\_64 floating point instructions
provided by the \texttt{Streaming SIMD Extensions 2} (SSE2) which operates on the \texttt{xmm}
registers. As SSE2 is part of x86\_64 it can be assuemd that SSE2 instructions are usable. But the
usage of other x86\_64 extensions like SSE4 or \texttt{Fused Multiply Add 3} (FMA3) have to
be allowed using a optimize command line flag.

\par

Due to performance reasons the occurring x86\_64 floating point exceptions are not transferred to
the variable representing the \emph{fcsr} which contain the floating points exceptions on RISC-V
processors. It is too slow to check after each operation which exceptions were raised and then set
the according flags. A read from the fcsr value can possibly be redirected to a read from the MXCSR
regsiter, but this would also take an implementation and performance overhead. However it is not a problem
because only very few programs depends on floating point exceptions.

\par

The floating point support can be deactivated using a command line flag in order to prevent the
lifter from adding the additional statics and to lift RISC-V floating point instructions. The
advantage is a decrease in used memory because less static variables have to be stored. The lifter
then ignores all floating point instructions instead of throwing an error because the ELF parser
cannot distinguish between instructions and constants as emphasized in Section
\ref{sec:program_decoding}. So it is avoided to throw an error if a constant can be decoded as an
instructions.

\subsection{Rounding and Accuracy Problems}

A known problem when dealing with different floating point implementations is that especially
rounding effects will lead to slightly different results. As we want to use the SSE2 floating point
instructions to provide an acceptable performance, some compromises have to be made.

\par

Firstly, for each RISC-V floating point instruction a rounding mode can be specified which is used
if the number cannot be expressed in the given number format. But as x86\_64 does not support such a
feature, it is quite hard to implement this in a fast way, because the results are already rounded
and therefore it is not possible to round the result of each operation afterwards.

\par

Also a bitwise correct implementation as provided by the dynamic translator \texttt{QEMU-user} is
challenging to implement and comparatively slow. So, we decided to ignore the rounding mode at lifting
every instruction except conversions between integers and floating points as well as between single
and double precision floating points. We made this exception as it makes a great difference when
rounding to integers. For example, the difference rounding \emph{25,5} towards positive infinity or
towards negative infinity, resulting in \emph{26} respectively \emph{25}, makes a huge difference.
Also the the later on used benchmarks (Section~\ref{sec:evaluation}) require to respect the rounding mode of
conversion instructions.

\par

There are two ways of implementing this specific rounding.

\begin{enumerate}
    \item The first one is to use the \texttt{roundss} instruction (respectively \texttt{roundsd}
          for double precision) and the according rounding mode as immediate input. But these
          instructions are part of SSE4 which is not implemented in every today used x86\_64-based
          CPU ~\cite{intel2017man}. Therefore the usage of SSE4 instructions must be allowed for this method.

    \item The alternative way is to change the rounding mode in the \emph{MXCSR} register. The
          x86\_64 convert instructions then round accordingly. But the disadvantage of this method
          is that the only way to access the MXCSR register is using \texttt{STMXCSR} and
          \texttt{LDMXCSR} instructions which only operates on memory operands and therefore is
          slower as the first method. Additionally this method uses more instructions in our
          implementation.~\cite{intel2017man}
\end{enumerate}

\par


A second problem is that there can be differences in rouding when assembling some instructions. For
example, the fused multiply add instructions can be translated using the x86\_64 FMA3 instructions.
But as this extension isn't supported by every x86\_64 CPU, it needs to be allowed. The alternative
way is to assemble the instruction using multiplication and addition respectively
subtraction~\footnote{For some fused mutliply adds, negating a floating point value is also
    needed.}. This can result in a different result due to rouding errors.


\subsection{Implementation Problems and Solutions}

Some RISC-V instructions does not have an x86\_64 equivalent, especially unsigned conversions, sign
injection and floating point classify. This instructions are handled by sequences of x86\_64
operations.

\par

The conversion from and to unsigned integers are implemented using the signed version. But the
result are corrected using some arithmetic. The conversions to unsigned integers are handled by
setting the result to zero if the input floating point is signed as specified by the RISC-V
manual~\cite{rvspec} using bit arithmetic. The other way around is solved by using the 64-bit
conversion for 32-bit inputs and for 64-bit inputs by handling according to the sign. If the value is
signed, the result of the signed conversion instruction needs to be adjusted by using bit arithmetic
as also used by gcc.

\par

We considered using the x86\_64 unsigned conversion instructions provided by the \texttt{Advanced
    Vector Extensions 512} (AVX512), but as AVX512 is relatively new and implemented in only few CPUs we
decided to do not provide an optimization flag for using this instructions.

\par

The sign injection and floating point classify instruction are implemented by using bit arithmetic
to manipulate the sign bit and to check in which category\footnote{For example, sNaN, qNaN, negative
    infinity, negative normal number, etc.} the floating point value fits, respectively.

\section{Evaluation}\label{sec:evaluation}

To measure the performance we used the SPEC CPU\textregistered 2017~\cite{spec_cpu_2017} benchmark package.
Because we did not optimize floating point operations we only ran the Integer and not the Floating Point suite.
However the Integer suite includes tests that not only cover integer instructions, but also many floating point operations, allowing us to verify our implementation.

\par

The integer suite covers many actual use cases, including compilers, video compression, artificial intelligence and others, allowing us to stress
test the performance under conditions similiar to a real situation.

\par

We compared the results to QEMU, a popular dynamic binary translator supporting many different architectures.
And also to RIA-JIT~\cite{ria_jit_repo}, another dynamic binary translator that only supports RISC-V to x86\_64 translation, but claims to be faster than QEMU in their paper~\cite{ria_jit_paper}.

\par

The effectiveness and drawbacks of the implemented optimizations were also measured.
% TODO: add thingy about fp accuracy, if we move more things here we need to rewrite this.

\subsection{Benchmark Setup}

All benchmarks where performaned on the \texttt{time-x} server provided by the Chair of Computer Architecture and Parallel Systems.~\footnote{\url{https://www.in.tum.de/caps/hw/caps-cloud/}} % FIXME: better citation
The Server has a \texttt{AMD Ryzen Threadripper PRO 3955WX 16-Cores} CPU and \texttt{32GB} memory.
It is running \texttt{Ubuntu 20.04.2 LTS} with kernel release \texttt{5.10.0-1044-oem} and version \texttt{\#46-Ubuntu SMP Wed Aug 11 09:50:57 UTC 2021}.
% TODO: mention vulnerability mitigations ? (lscpu)
The benchmarks where executed on a \texttt{nfs} filesystem.

For native benchmarks the GNU Compiler provided by the operating system was used,
version \texttt{9.3.0 (Ubuntu 9.3.0-17ubuntu1~20.04)}, and glibc version \texttt{2.31-0ubuntu9.2}.
For RISC-V benchmarks the gnu toolchain provided by the RISC-V foundation is used, version 2021.09.21~\footnote{Repository \url{https://github.com/riscv-collab/riscv-gnu-toolchain}, Commit hash b39e36160aa0649ba0dfb9aa314d375900d610fb},
it was configured with \texttt{--with-arch=rv64imafdc --with-abi=lp64d}.

\par

Our own translator was build from the latest master~\footnote{Commit hash 3c341ea7298555b7c34572bd355ca4765d7650df} at the time in release mode.
The QEMU version used was \texttt{1:4.2-3ubuntu6.17}.
For ria-jit a modified version by Alexis Engelke was used, adding support for RV64C over the original implementation.
It was further modified to fix a bug with compressed instructions and to implement an additional system call required by the benchmarks.
~\footnote{Commit hash cff2c15e4b646dca8d839f4a5ee9c3ec89231ac6}

\par

Due to lack of development tools, the toolchain, ria and sbt where compiled in a container similiar to \texttt{time-x} on another host and uploaded. % FIXME: remove because not useful ?

% TODO: ria-jit version (commit hash + changes to original paper, cite paper and reason we choose to compare against it, maybe "because it is faster than qemu" ?)
% TODO: describe changes to the benchmark configuration for compiler compatibility, see `diff config/Example... config/ourconfig.cfg` in time-x for details, ignore the __HASH__ things
% TODO: version of SPEC CPU 2017 used: SPEC CPU(R)2017 v1.1.0
% TODO: note that for our translator only the execution time of the resulting binary was measured, not of the translation step

\subsection{Correctness}

To verify that our translator produces correct binaries we used the testsuites of the SPEC CPU 2017 benchmarks.
Especially the tests for the \texttt{600.perlbench} benchmark are ideal for this purpose, as they they test the perl interpreter and some popular perl packages.
As a result they cover a wide range of operations, including many different system calls, floating point and string operations.
Other benchmark tests cover compilation and compression.
All together the tests cover many real use cases, any binary passing can be assumed to be correct in almost all cases.

\par

Using this methology we tested that the interpreter was implemented correctly.
The translator also passes all tests when compiled without floating point support.
However with floating point support additional optimizations~\footnote{\texttt{fma3} and \texttt{sse4} need to be enabled.} need to be enabled to achieve correct floating point results.
% TODO: either say they need fixing or that this is due to wrong rounding (backref to floating chapter)

\subsection{Comparison to QEMU and RIA-JIT}

\begin{figure}
    \begin{centering}
        \begin{tikzpicture}
            \begin{axis}[
                    width=0.9\textwidth,
                    height=0.5\textwidth,
                    ybar,
                    bar width = 4pt,
                    ylabel={Execution time normalised to native (less is better)},
                    symbolic x coords = {
                            600.perlbench,
                            602.gcc,
                            605.mcf,
                            620.omnetpp,
                            623.xalancbmk,
                            625.x264,
                            631.deepsjeng,
                            641.leela,
                            648.exchange2,
                            657.xz,
                            average % TODO: find a way to automatically compute this row
                        },
                    xticklabel style = {
                            anchor = east,
                            rotate = 45 + 15
                        },
                    xtick = data, % make a tick at every symbolic x coordinate
                    ytick distance = 1, % make a tick every 1
                    ymajorgrids = true, % display horizontal guide lines
                    legend style = { at={(0.95, 0.95)}, anchor=north east}, % place the graph legend at the top-right
                ]

                \addplot[fill=green]  table[x=benchmark, y expr=\thisrow{native} /\thisrow{native}] from {./tikz_src/benchmarks_results1.txt};
                \addplot[fill=orange] table[x=benchmark, y expr=\thisrow{qemu}   /\thisrow{native}] from {./tikz_src/benchmarks_results1.txt};
                \addplot[fill=yellow] table[x=benchmark, y expr=\thisrow{ria}    /\thisrow{native}] from {./tikz_src/benchmarks_results1.txt};
                % \addplot[fill=red]    table[x=benchmark, y expr=\thisrow{master} /\thisrow{native}] from {./tikz_src/benchmarks_results1.txt};
                \addplot[fill=blue]    table[x=benchmark, y expr=\thisrow{sbt1}   /\thisrow{native}] from {./tikz_src/benchmarks_results1.txt};
                %\addplot[fill=blue]    table[x=benchmark, y expr=\thisrow{test3}  /\thisrow{native}] from {./tikz_src/benchmarks_results1.txt};
                %                \addplot table[x=benchmark, y expr=\thisrow{test2} /\thisrow{native}] from {./tikz_src/benchmarks_results1.txt};
                %                \addplot table[x=benchmark, y expr=\thisrow{ria}   /\thisrow{native}] from {./tikz_src/benchmarks_results1.txt};

                %                \legend{native, qemu, test1, test2, ria}
                \legend{
                    native,
                    qemu,
                    ria,
                    % master,
                    sbt1}
            \end{axis}
        \end{tikzpicture}
        \caption{
            SPEC CPU 2017 Results Compared to SBT executed with all optimizations except \texttt{no\_trans\_bbs} enabled.
            RIA-JIT encountered a floating point error in the benchmark 625.x264, so this single benchmark was repeated with a toolchain using \texttt{--with-arch=rv64iamc --with-abi=lp64}.
        }\label{benchmark_results1}
    \end{centering}
\end{figure}

\ref{benchmark_results1}

Due to bugs in the ria-jit rv64fd implementation, namely the lack of dynamic rounding support % TODO: verify
the xz benchmark did not complete successfully.

\subsection{Effectiveness of different Optimizations}

\subsection{Basic Block Start Address Hashing}

\begin{figure}
    \begin{centering}
        \begin{tikzpicture}
            \begin{axis}[
                    width=0.9\textwidth,
                    height=0.5\textwidth,
                    tick align=inside,
                    unbounded coords=jump,
                    xmin=0,
                    xmax=110,
                    ymin=-1,
                    ymax=22,
                    point meta min=90,
                    point meta max=108,
                    colorbar horizontal,
                    colorbar style={
                            xlabel={Assembly file size in $\frac{1}{100}\cdot \textit{Binary size without hashing}$},
                        },
                    colormap={reverse viridis}{
                            indices of colormap={\pgfplotscolormaplastindexof{viridis},...,0 of viridis}
                        },
                    xlabel={Load factor in percent},
                    ylabel={Average bucket size in $\frac{1}{\text{Bucket}}$},
                ]
                \addplot[
                    mark=square*,
                    only marks,
                    scatter,
                    scatter src=explicit,
                    mark size=3,
                ]
                table[meta=z] from {./tikz_src/hashing_assembly_size.txt};
            \end{axis}
        \end{tikzpicture}
        \caption{Effects of average bucket size and hash table load factor on final assembly file size.}\label{hashing_figure}
    \end{centering}
\end{figure}

In Figure~\ref{hashing_figure} we show the effects of the two major parameters for the used hashing
algorithm. Missing data points mean that the hash function algorithm was not able to find all valid
bucket-level hash functions within the 16-bit per-bucket storage limit. The load factor is the fraction (in percent) of the
final hash table which is filled with
hashed entries. The average bucket size is used to calculate the number of buckets and therefore
directly effects the number of bucket-level hash functions required.

\par

The test was conducted on the translated binary of a simple zip program. Sources are located in the
\texttt{tests} project folder. The test was only performed once because the hashing algorithm and
the translator itself are both strictly deterministic.

\par

The analysis shows that
the amount of storage required for bucket-level hash functions does not have a noticeable impact on
the final assembly size. It also shows that finding all valid bucket-level hash functions is more
likely with a lower average number of elements per bucket. This leads to the conclusion that using
smaller buckets sizes can improve bucket-level hash function generation times while not noticeably
increasing the output assembly size.

\subsection{Performance of Translation Blocks}

\subsection{Floatingpoint accuracy}

\begin{figure}[H]
    \centering
    \includegraphics[width=0.75\textwidth]{images/mandelbrot_differences/translated_diff.png}
    \caption{Differences in results of the mandelbrot set visualizer~\cite{mandelbrot_program} when
        comparing the translator to RISC-V emulated by QEMU.}\label{fig:mandelbrot_diff_translator}
\end{figure}

To depict these rounding differences we use a simple mandelbrot set
visualizer~\cite{mandelbrot_program}. The results provided in a textual form are transformed into a
picture coloring the mandelbrot set in a light grey, differences between the implementations with
red and the remainder in a dark grey. Figure~\ref{fig:mandelbrot_diff_translator} shows that the
translator without any optimizations produces different results than the RISC-V binary emulated with
QEMU. Tests also showed that the native x86\_64 compiled version leads to the same picture.

\par

\begin{figure}[H]
    \centering
    \includegraphics[width=0.75\textwidth]{images/mandelbrot_differences/fma_diff.png}
    \caption{Differences in results of the mandelbrot set visualizer~\cite{mandelbrot_program} when
        comparing the translator with all optimizations (especially FMA3) to RISC-V.}\label{fig:mandelbrot_diff_fma3}
\end{figure}

Another problem can be observed when activating the usage of FMA3 and therefore allows the
translation of RISC-V fused multiply add instruction to x86\_64 equivalents rather than assembling
it. This again has an impact on the accuracy, because the intermediate result in fused multiply add
instructions is not rounded~\cite{intel2017man}. But as shown by
Figure~\ref{fig:mandelbrot_diff_fma3} this causes the result to be more similar to the RISC-V result
than the assembled version.

\section{Summary}

\clearpage

\bibliographystyle{plain}
\bibliography{Ausarbeitung}{}

\appendix
\section{Appendix}

\begin{center}
    \begin{longtable}{p{0.35\linewidth} | p{0.6\linewidth}}
        \hline
        Instruction  & Description                                                                                                                \\ [0.5ex]
        \hline
        store        & Store value at supplied address. Consumes memory token and generates a new one.                                            \\
        \hline
        load         & Loads value at supplied address and returns it. Reads as many bits from memory as the output
        variable is wide.                                                                                                                         \\
        \hline
        add          & Addition.                                                                                                                  \\
        \hline
        sub          & Subtraction.                                                                                                               \\
        \hline
        mul\_l       & Multiplication. Returns the lower half of the result.                                                                      \\
        \hline
        ssmul\_h     & \textbf{Signed} multiplication. Returns upper half of the result.                                                          \\
        \hline
        uumul\_h     & \textbf{Unsigned} multiplication. Returns upper half of the result.                                                        \\
        \hline
        sumul\_h     & Multiplication of one \textbf{signed} and one \textbf{unsigned} value. Returns upper half of the result.                   \\
        \hline
        div          & \textbf{Signed} division. Returns both the result and the remainder.                                                       \\
        \hline
        udiv         & \textbf{Unsigned} division. Returns both the result and the remainder.                                                     \\
        \hline
        shl          & Logical shift left.                                                                                                        \\
        \hline
        shr          & Logical shift right.                                                                                                       \\
        \hline
        sar          & Arithmetic shift right.                                                                                                    \\
        \hline
        or           & Logical \textit{or}.                                                                                                       \\
        \hline
        and          & Logical \textit{and}.                                                                                                      \\
        \hline
        not          & Logical \textit{not}.                                                                                                      \\
        \hline
        xor          & Logical \textit{xor}.                                                                                                      \\
        \hline
        cast         & C-style typecast. Defined for shrinking variables and for bit-wise cast from integer to
        floating point numbers.                                                                                                                   \\
        \hline
        slt          & Ternary operator, set if less than:\newline IR: \texttt{dst $\leftarrow$ slt v1, v2, v3, v4} \newline
        Pseudocode: \texttt{dst = (v1 < v2) ? v3 : v4}                                                                                            \\
        \hline
        sltu         & Ternary operator, set if less than:\newline IR: \texttt{dst $\leftarrow$ sltu v1, v2, v3, v4} \newline
        Pseudocode: \texttt{dst = (v1 <\textbf{u} v2) ? v3 : v4}                                                                                  \\
        \hline
        sle          & Ternary operator, set if less or equal:\newline IR: \texttt{dst $\leftarrow$ sle v1, v2, v3, v4} \newline
        Pseudocode: \texttt{dst = (v1 <= v2) ? v3 : v4}                                                                                           \\
        \hline
        seq          & Ternary operator, set if equal:\newline IR: \texttt{dst $\leftarrow$ seq v1, v2, v3, v4} \newline
        Pseudocode: \texttt{dst = (v1 == v2) ? v3 : v4}                                                                                           \\
        \hline
        sign\_extend & Enlarge value to bigger type using sign extension.                                                                         \\
        \hline
        zero\_extend & Enlarge value to bigger type using zero extension.                                                                         \\
        \hline
        setup\_stack & Meta instruction used for stack setup at program start.                                                                    \\
        \hline
        umax         & Returns larger one of two values, \textbf{unsigned} comparison.                                                            \\
        \hline
        umin         & Returns smaller one of two values, \textbf{unsigned} comparison.                                                           \\
        \hline
        max          & Returns larger one of two values, \textbf{signed} comparison.                                                              \\
        \hline
        min          & Returns smaller one of two values, \textbf{signed} comparison.                                                             \\
        \hline
        fmul         & \textbf{Signed} floating pointer number multiplication.                                                                    \\
        \hline
        fdiv         & \textbf{Signed} floating pointer number division, only returns the result.                                                 \\
        \hline
        fsqrt        & Takes the square root of two floating point values.                                                                        \\
        \hline
        fmadd        & Fused multiply add of three floating point values:\newline IR: \texttt{dst $\leftarrow$ fmadd v1, v2, v3} \newline
        Pseudocode: \texttt{dst = v1 * v2 + v3}                                                                                                   \\
        \hline
        fmsub        & Fused multiply sub of three floating point values:\newline IR: \texttt{dst $\leftarrow$ fmsub v1, v2, v3} \newline
        Pseudocode: \texttt{dst = v1 * v2 - v3}                                                                                                   \\
        \hline
        fnmadd       & Fused negate multiply add of three floating point values:\newline IR: \texttt{dst $\leftarrow$ fnmadd v1, v2, v3} \newline
        Pseudocode: \texttt{dst = -(v1 * v2) + v3}                                                                                                \\
        \hline
        fnmsub       & Fused negate multiply sub of three floating point values:\newline IR: \texttt{dst $\leftarrow$ fnmsub v1, v2, v3} \newline
        Pseudocode: \texttt{dst = -(v1 * v2) - v3}                                                                                                \\
        \hline
        convert      & Conversion between integer and floating point numbers or between single and double
        precision floating point numbers.
        \\
        \hline
        uconvert     & Unsigned conversion between integer and floating point numbers or between single and double
        precision floating point numbers.                                                                                                         \\
        \hline
    \end{longtable}
    \captionof{figure}{Defined IR instructions.}\label{figure:ir_instructions}
\end{center}
\end{document}
