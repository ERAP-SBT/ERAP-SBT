%%%%%%%%%%%%%%%%%%%%%%%%%%%%%%%%%%%%%%%%%%%%%%%%%%%%%%%%%%%%%%%%%%%%%%%%%%%%%%%%
% TUM-Vorlage: Präsentation
%%%%%%%%%%%%%%%%%%%%%%%%%%%%%%%%%%%%%%%%%%%%%%%%%%%%%%%%%%%%%%%%%%%%%%%%%%%%%%%%
%
% Rechteinhaber:
%     Technische Universität München
%     https://www.tum.de
% 
% Gestaltung:
%     ediundsepp Gestaltungsgesellschaft, München
%     http://www.ediundsepp.de
% 
% Technische Umsetzung:
%     eWorks GmbH, Frankfurt am Main
%     http://www.eworks.de
%
%%%%%%%%%%%%%%%%%%%%%%%%%%%%%%%%%%%%%%%%%%%%%%%%%%%%%%%%%%%%%%%%%%%%%%%%%%%%%%%%


%%%%%%%%%%%%%%%%%%%%%%%%%%%%%%%%%%%%%%%%%%%%%%%%%%%%%%%%%%%%%%%%%%%%%%%%%%%%%%%%
% Zur Wahl des Seitenverhältnisses bitte einen der beiden folgenden Befehle
% auskommentieren und den ausführen lassen:
\input{praeambel_16_9.tex} % Seitenverhältnis 16:9
%%%%%%%%%%%%%%%%%%%%%%%%%%%%%%%%%%%%%%%%%%%%%%%%%%%%%%%%%%%%%%%%%%%%%%%%%%%%%%%%


%%%%%%%%%%%%%%%%%%%%%%%%%%%%%%%%%%%%%%%%%%%%%%%%%%%%%%%%%%%%%%%%%%%%%%%%%%%%%%%%
%%%%%%%%%%%%%%%%%%%%%%%%%%%%%%%%%%%%%%%%%%%%%%%%%%%%%%%%%%%%%%%%%%%%%%%%%%%%%%%%
% TUM-Vorlage: Personenspezifische Informationen
%%%%%%%%%%%%%%%%%%%%%%%%%%%%%%%%%%%%%%%%%%%%%%%%%%%%%%%%%%%%%%%%%%%%%%%%%%%%%%%%
%
% Rechteinhaber:
%     Technische Universität München
%     https://www.tum.de
% 
% Gestaltung:
%     ediundsepp Gestaltungsgesellschaft, München
%     http://www.ediundsepp.de
% 
% Technische Umsetzung:
%     eWorks GmbH, Frankfurt am Main
%     http://www.eworks.de
%
%%%%%%%%%%%%%%%%%%%%%%%%%%%%%%%%%%%%%%%%%%%%%%%%%%%%%%%%%%%%%%%%%%%%%%%%%%%%%%%%

% Für die Person anpassen:

% Allgemein:
\newcommand{\AllgemeinGestalter}{ediundsepp Gestaltungsgesellschaft}
\newcommand{\AllgemeinErsteller}{eWorks GmbH}

% Universität:
\newcommand{\UniversitaetName}{Technische Universität München}
\newcommand{\UniversitaetAbkuerzung}{TUM}
\newcommand{\UniversitaetWebseite}{www.tum.de}
\newcommand{\UniversitaetLogoBreite}{19mm}
\newcommand{\UniversitaetLogoHoehe}{1cm}

\newcommand{\UniversitaetAdresse}{%
	Arcisstraße~21\\%
	80333~München%
}

\hyphenation{} % eigene Silbentrennung
                    % !!! DATEI ANPASSEN !!!
%%%%%%%%%%%%%%%%%%%%%%%%%%%%%%%%%%%%%%%%%%%%%%%%%%%%%%%%%%%%%%%%%%%%%%%%%%%%%%%%

\newcommand{\Datum}{\today}

\renewcommand{\PraesentationFusszeileZusatz}{Rechnerarchitektur-Großpraktikum 2021 | Statische Binärübersetzung von RISC-V in x86-64}

\title{Statische Binärübersetzung von RISC-V in x86-64}
\author{Lukas Döllerer, Jonathan Hettwer, Johannes Maier, Tobias Schwarz, Felix Solcher}
\institute[]{Rechnerarchitektur-Großpraktikum 2021}
\date[\Datum]{Garching, 16. Juli 2021}
\subject{Statische Binärübersetzung von RISC-V in x86-64}


%%%%%%%%%%%%%%%%%%%%%%%%%%%%%%%%%%%%%%%%%%%%%%%%%%%%%%%%%%%%%%%%%%%%%%%%%%%%%%%%
\input{entry.tex} % !!! NICHT ENTFERNEN !!!
%%%%%%%%%%%%%%%%%%%%%%%%%%%%%%%%%%%%%%%%%%%%%%%%%%%%%%%%%%%%%%%%%%%%%%%%%%%%%%%%


%%%%%%%%%%%%%%%%%%%%%%%%%%%%%%%%%%%%%%%%%%%%%%%%%%%%%%%%%%%%%%%%%%%%%%%%%%%%%%%%
% FOLIENSTIL: Standard
\PraesentationMasterStandard

\PraesentationTitelseite % Fügt die Startseite ein

% TODO: Create Program Flow Diagram

%draws an arrow from (#2,#3) to (#2+#4,#3) with height of #5 (arrow in x direction)
%color, startX, startY, length (X dir), height (Y dir)
\newcommand{\TikZArrowX}[5]{\draw[#1] (#2,#3) -- (#2,#3+#5*2/3) -- (#2+#4/2,#3+#5*2/3) -- (#2+#4/2,#3+#5) -- (#2+#4,#3) -- (#2+#4/2,#3-#5) -- (#2+#4/2,#3-#5*2/3) -- (#2,#3-#5*2/3) -- (#2,#3);}

%draws an arrow from #2,#3) to (#2,#3+#5) with height of #4 (arrow in y direction)
%color, startX, startY, length (X dir), height (Y dir)
\newcommand{\TikZArrowY}[5]{\draw[#1] (#2,#3) -- (#2+#4*2/3,#3) -- (#2+#4*2/3,#3+#5/2) -- (#2+#4,#3+#5/2) -- (#2,#3+#5) -- (#2-#4,#3+#5/2) -- (#2-#4*2/3,#3+#5/2) -- (#2-#4*2/3,#3) -- (#2,#3);}

%draws the schematic presentation of the program
%scale, fontSize
\newcommand{\ProgramSchematic}[2]{
    \begin{center}
        \begin{tikzpicture}[very thick, scale=#1]
            \draw[TUMBlau] (-2,-2) rectangle (2,2);
            \node[align=center] at (0,0) (riscv_text) {\fontsize{#2}{#2} \selectfont RISC-V};
            \TikZArrowX{TUMBlau}{2}{0}{6}{1.5}
            \node[align=center] at (4.5,0) (lifter_text) {\fontsize{#2}{#2} \selectfont Lifter};
            \draw[TUMBlau] (8,-2) rectangle (12,2);
            \node[align=center] at (10,0) (ir_text_1) {\fontsize{#2}{#2} \selectfont IR};
            \TikZArrowX{TUMBlau}{12}{0}{6}{1.5}
            \node[align=center] at (14.5,0) (optimizer_text) {\fontsize{#2}{#2} \selectfont Optimizer};
            \draw[TUMBlau] (18,-2) rectangle (22, 2);
            \node[align=center] at (20,0) (ir_text_2) {\fontsize{#2}{#2} \selectfont IR};
            \TikZArrowX{TUMBlau}{22}{0}{6}{1.5}
            \node[align=center] at (24.5,0) (compiler_text) {\fontsize{#2}{#2} \selectfont Compiler};
            \draw[TUMBlau] (28,-2) rectangle (32,2);
            \node[align=center] at (30,0) (x86_64_text) {\fontsize{#2}{#2} \selectfont x86\_64};
        \end{tikzpicture}
    \end{center}
}

\begin{frame}
    \frametitle{Programmübersicht}
    %alignment to have some space between headline an the schematic
    ~\\
    ~\\
    \ProgramSchematic{0.6}{18}
\end{frame}


%%%%%%%%%%%%%%%%%%%%%%%%%%%%%%%%%%%%%%%%%%%%%%%%%%%%%%%%%%%%%%%%%%%%%%%%%%%%%%%%
\end{document} % !!! NICHT ENTFERNEN !!!
%%%%%%%%%%%%%%%%%%%%%%%%%%%%%%%%%%%%%%%%%%%%%%%%%%%%%%%%%%%%%%%%%%%%%%%%%%%%%%%%

