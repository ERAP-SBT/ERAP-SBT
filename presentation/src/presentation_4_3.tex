%%%%%%%%%%%%%%%%%%%%%%%%%%%%%%%%%%%%%%%%%%%%%%%%%%%%%%%%%%%%%%%%%%%%%%%%%%%%%%%%
% TUM-Vorlage: Präsentation
%%%%%%%%%%%%%%%%%%%%%%%%%%%%%%%%%%%%%%%%%%%%%%%%%%%%%%%%%%%%%%%%%%%%%%%%%%%%%%%%
%
% Rechteinhaber:
%     Technische Universität München
%     https://www.tum.de
%
% Gestaltung:
%     ediundsepp Gestaltungsgesellschaft, München
%     http://www.ediundsepp.de
%
% Technische Umsetzung:
%     eWorks GmbH, Frankfurt am Main
%     http://www.eworks.de
%
%%%%%%%%%%%%%%%%%%%%%%%%%%%%%%%%%%%%%%%%%%%%%%%%%%%%%%%%%%%%%%%%%%%%%%%%%%%%%%%%


%%%%%%%%%%%%%%%%%%%%%%%%%%%%%%%%%%%%%%%%%%%%%%%%%%%%%%%%%%%%%%%%%%%%%%%%%%%%%%%%
% Zur Wahl des Seitenverhältnisses bitte einen der beiden folgenden Befehle
% auskommentieren und den ausführen lassen:
\documentclass[handout,t,aspectratio=43]{beamer}
% \documentclass[notes,t,aspectratio=43]{beamer}
\usepackage[
    orientation=landscape,
    size=custom,
    width=25.4,
    height=19.05,
    scale=0.7 % erzeugt 16pt Schriftgröße
]{beamerposter}

% Display Notes for pympress
\usepackage{pgfpages}
% \setbeameroption{show notes on second screen=right}
\setbeamertemplate{note page}[plain]
\setbeamerfont{note page}{size=\Large}

\newenvironment{itemize*}%
  {\begin{itemize}%
    \setlength{\itemsep}{5pt}%
    \setlength{\parskip}{0pt}}%
  {\end{itemize}}

\newcommand{\PraesentationSchriftgroesseSehrGross}{\fontsize{30}{45}}
\newcommand{\PraesentationSchriftgroesseGross}{\fontsize{22}{33}}
\newcommand{\PraesentationSchriftgroesseNormal}{\fontsize{16}{29}}
\newcommand{\PraesentationSchriftgroesseKlein}{\fontsize{12}{18}}
\newcommand{\PraesentationSchriftgroesseDreizeiler}{\fontsize{7}{10}}
\newcommand{\PraesentationSchriftgroesseAufzaehlungszeichen}{\fontsize{10}{8}}

\newcommand{\PraesentationAbstandAbsatz}{22.1pt}
\newcommand{\PraesentationPositionKorrekturOben}{0cm}
\newcommand{\PraesentationBeispieleSchriftgroessen}{30 | 22 | 16 | 12}
\input{praeambel.tex}
 % Seitenverhältnis 4:3
%%%%%%%%%%%%%%%%%%%%%%%%%%%%%%%%%%%%%%%%%%%%%%%%%%%%%%%%%%%%%%%%%%%%%%%%%%%%%%%%


%%%%%%%%%%%%%%%%%%%%%%%%%%%%%%%%%%%%%%%%%%%%%%%%%%%%%%%%%%%%%%%%%%%%%%%%%%%%%%%%
%%%%%%%%%%%%%%%%%%%%%%%%%%%%%%%%%%%%%%%%%%%%%%%%%%%%%%%%%%%%%%%%%%%%%%%%%%%%%%%%
% TUM-Vorlage: Personenspezifische Informationen
%%%%%%%%%%%%%%%%%%%%%%%%%%%%%%%%%%%%%%%%%%%%%%%%%%%%%%%%%%%%%%%%%%%%%%%%%%%%%%%%
%
% Rechteinhaber:
%     Technische Universität München
%     https://www.tum.de
% 
% Gestaltung:
%     ediundsepp Gestaltungsgesellschaft, München
%     http://www.ediundsepp.de
% 
% Technische Umsetzung:
%     eWorks GmbH, Frankfurt am Main
%     http://www.eworks.de
%
%%%%%%%%%%%%%%%%%%%%%%%%%%%%%%%%%%%%%%%%%%%%%%%%%%%%%%%%%%%%%%%%%%%%%%%%%%%%%%%%

% Für die Person anpassen:

% Allgemein:
\newcommand{\AllgemeinGestalter}{ediundsepp Gestaltungsgesellschaft}
\newcommand{\AllgemeinErsteller}{eWorks GmbH}

% Universität:
\newcommand{\UniversitaetName}{Technische Universität München}
\newcommand{\UniversitaetAbkuerzung}{TUM}
\newcommand{\UniversitaetWebseite}{www.tum.de}
\newcommand{\UniversitaetLogoBreite}{19mm}
\newcommand{\UniversitaetLogoHoehe}{1cm}

\newcommand{\UniversitaetAdresse}{%
	Arcisstraße~21\\%
	80333~München%
}

\hyphenation{} % eigene Silbentrennung
                    % !!! DATEI ANPASSEN !!!
%%%%%%%%%%%%%%%%%%%%%%%%%%%%%%%%%%%%%%%%%%%%%%%%%%%%%%%%%%%%%%%%%%%%%%%%%%%%%%%%

\newcommand{\Datum}{\today}

\renewcommand{\PraesentationFusszeileZusatz}{Rechnerarchitektur-Großpraktikum 2021 | Statische Binärübersetzung von RISC-V in x86-64}

\title{Statische Binärübersetzung von RISC-V in x86-64}
\author{Lukas Döllerer, Jonathan Hettwer, Johannes Maier, Tobias Schwarz, Felix Solcher}
\institute[]{Rechnerarchitektur-Großpraktikum 2021}
\date[\Datum]{Garching, 16. Juli 2021}
\subject{Statische Binärübersetzung von RISC-V in x86-64}


%%%%%%%%%%%%%%%%%%%%%%%%%%%%%%%%%%%%%%%%%%%%%%%%%%%%%%%%%%%%%%%%%%%%%%%%%%%%%%%%
\input{entry.tex} % !!! NICHT ENTFERNEN !!!
%%%%%%%%%%%%%%%%%%%%%%%%%%%%%%%%%%%%%%%%%%%%%%%%%%%%%%%%%%%%%%%%%%%%%%%%%%%%%%%%


%%%%%%%%%%%%%%%%%%%%%%%%%%%%%%%%%%%%%%%%%%%%%%%%%%%%%%%%%%%%%%%%%%%%%%%%%%%%%%%%
% FOLIENSTIL: Standard
\PraesentationMasterStandard

\PraesentationTitelseite % Fügt die Startseite ein

% draws an arrow from (#2,#3) to (#2+#4,#3) with height of #5 (arrow in x direction)
% color (and other arguments, like visible on), startX, startY, length (X dir), height (Y dir)
\newcommand{\TikZArrowX}[5]{
    \filldraw[#1] (#2,#3) -- (#2,#3+#5*2/3) -- (#2+#4/2,#3+#5*2/3) -- (#2+#4/2,#3+#5) -- (#2+#4,#3) -- (#2+#4/2,#3-#5) -- (#2+#4/2,#3-#5*2/3) -- (#2,#3-#5*2/3) -- (#2,#3);
    \draw[#1, black] (#2,#3) -- (#2,#3+#5*2/3) -- (#2+#4/2,#3+#5*2/3) -- (#2+#4/2,#3+#5) -- (#2+#4,#3) -- (#2+#4/2,#3-#5) -- (#2+#4/2,#3-#5*2/3) -- (#2,#3-#5*2/3) -- (#2,#3);
}

% draws an arrow from #2,#3) to (#2,#3+#5) with height of #4 (arrow in y direction)
% color (and other arguments, like visible on), startX, startY, length (X dir), height (Y dir), color
\newcommand{\TikZArrowY}[5]{
    \filldraw[#1] (#2,#3) -- (#2+#4*2/3,#3) -- (#2+#4*2/3,#3+#5/2) -- (#2+#4,#3+#5/2) -- (#2,#3+#5) -- (#2-#4,#3+#5/2) -- (#2-#4*2/3,#3+#5/2) -- (#2-#4*2/3,#3) -- (#2,#3);
    \draw[#1, black] (#2,#3) -- (#2+#4*2/3,#3) -- (#2+#4*2/3,#3+#5/2) -- (#2+#4,#3+#5/2) -- (#2,#3+#5) -- (#2-#4,#3+#5/2) -- (#2-#4*2/3,#3+#5/2) -- (#2-#4*2/3,#3) -- (#2,#3);
}

% draws one entry of the color legend for the program scheme
% color, text, posX, posY (left, top coordinate)
\newcommand{\colorLegendEntry}[4]{
    \filldraw[#1] (#3,#4) rectangle (#3+1,#4+1);
    \draw[black] (#3,#4) rectangle (#3+1,#4+1);
    \node at (#3+0.5,#4+0.4) (color_legend_entry_point) {};
    \node[right=2mm of color_legend_entry_point] (color_legend_entry_text) {#2};
}

% draws a color legend for the program scheme
% posX, posY (left, top coordinate)
\newcommand{\colorLegend}[2] {
    \colorLegendEntry{black!30!orange}{Static translator parts}{#1}{#2}
    \colorLegendEntry{blue}{Intermediate representation (IR)}{#1}{#2-1.5}
    \colorLegendEntry{purple}{Machine code / ELF file}{#1+20}{#2}
}

% draws the schematic presentation of the program
% scale, fontSize
\newcommand{\ProgramScheme}[2]{
    \begin{tikzpicture}[scale=#1]
        % draw color legend
        \colorLegend{-1}{-6}

        % translator rectangle
        \draw[black, thick] (2.5,4) rectangle (27.5, -4);
        \node[align=center, color=black] at (15, 3) (translator_text) {\fontsize{#2}{#2} \selectfont Translator};

        % riscv elf file rectangle
        \draw[black, fill=purple] (-2,-2) rectangle (2,2);
        % label
        \node[align=center, color=white] at (0,0) (riscv_text) {\fontsize{#2}{#2} \selectfont RISC-V \\ \fontsize{#2}{#2} \selectfont Binary};

        % lifter arrow
        \TikZArrowX{black!30!orange}{2}{0}{6}{1.5}
        % arrow label
        \node[align=center, color=white] at (4.5,0) (lifter_text) {\fontsize{#2}{#2} \selectfont Lifter};

        % ir (unoptimized) rectangle
        \draw[black, fill=blue] (8,-2) rectangle (12,2);
        % label
        \node[align=center, color=white] at (10,0) (ir_text_1) {\fontsize{#2}{#2} \selectfont IR};

        % optimizer arrow
        \TikZArrowX{black!30!orange}{12}{0}{6}{1.5}
        % arrow label
        \node[align=center, color=white] at (14.5,0) (optimizer_text) {\fontsize{#2}{#2} \selectfont Optimizer};

        % ir (optimized) rectangle
        \draw[black, fill=blue] (18,-2) rectangle (22, 2);
        % label
        \node[align=center, color=white] at (20,0) (ir_text_2) {\fontsize{#2}{#2} \selectfont IR};

        % compiler arrow
        \TikZArrowX{black!30!orange}{22}{0}{6}{1.5}
        % arrow label
        \node[align=center, color=white] at (24.5,0) (generator_text) {\fontsize{#2}{#2} \selectfont Generator};

        % x86_64 rectangle
        \draw[black, fill=purple] (28.5,-2) rectangle (32.5,2);
        % label
        \node[align=center, color=white] at (30.5,0) (x86_64_text) {\fontsize{#2}{#2} \selectfont Trans- \\ \fontsize{#2}{#2} \selectfont lated \\ \fontsize{#2}{#2} \selectfont Binary};

        % helper rectangle
        \draw[black, fill=purple] (28.5, 9) rectangle (32.5, 5);
        % helper label
        \node[align=center, color=white] at (30.5, 7) (helper_text) {\fontsize{#2}{#2} \selectfont Helper \\ \fontsize{#2}{#2} \selectfont Library};

        % connection between x86_64 and helper library
        \draw[black, -{Stealth[scale=3]}] (30, 2) -- (30, 5);
        \draw[black, {Stealth[scale=3]}-] (31, 2) -- (31, 5);

        % result binary rectangle
        \draw[black, thick] (28, 12) rectangle (33, -4);
        % result binary label
        \node[align=center, color=black] at (30.5, 10.5) (result_binary_text) {\fontsize{#2}{#2} \selectfont Result \\ \fontsize{#2}{#2} \selectfont Binary};
        
        % memory image arrow + label
        \draw[black, -{Stealth[scale=3]}] (0, 2) -- (0, 7) -- (28, 7);
        \node[align=center, color=black] at (15, 8) {\fontsize{#2}{#2} \selectfont Memory Image};
    \end{tikzpicture}
}


%%%%%%%%%%%%
%% Inhalt %%
%%%%%%%%%%%%

\begin{frame}
    \frametitle{Inhalt}
    \begin{itemize}
        \item
    \end{itemize}
\end{frame}

\begin{frame}
    \frametitle{Dynamische und Statische Binärübersetzung}{}

    \note[item]{Ziel: Statische Binärübersetzung}
    \note[item]{Übersetzung von Binärdateien auf eine andere Plattform}
    \note[item]{hier: von RISC-V zu x86\_64}
    \note[item]{Dynamisch wäre ein Interpreter, also zur Laufzeit, bei uns Übersetzung zur Compilezeit}

    \vspace{2em}
    Gemeinsames Ziel: Ausführung von Binärdateien einer Architektur auf einer anderen Architektur.
    \pause

    \vspace{1.25cm}

    \begin{columns}[t]
        \column{0.45\textwidth}
        \textbf{Dynamische Übersetzung}
        \begin{itemize}
            \vspace{1em}
            \setlength{\itemsep}{1em}
            \item Instruktionen werden zur Laufzeit übersetzt
            \item Übersetzte Instruktionen können zwischengespeichert werden
        \end{itemize}

        \column{0.5\textwidth}
        \textbf{Statische Übersetzung}
        \begin{itemize}
            \vspace{1em}
            \setlength{\itemsep}{1em}
            \item Binärdatei wird erst zur Zielarchitektur übersetzt
            \item Resultat kann unabhängig von Übersetzer ausgeführt werden
        \end{itemize}
    \end{columns}
\end{frame}

\begin{frame}{RISC-V Übersicht}
    \note[item]{Load-Store: Spezielle Befehle für Laden/Speichern in Register (-> Register-Register Architektur)}
    \vspace{1.25cm}
    \begin{itemize}
        \setlength\itemsep{0.8em}
        \item Relativ neuer Befehlssatz
        \item Aufgeteilt in Base und Extensions
        \item Feste Befehlslänge von 32-Bit (optional auch 16-Bit)
        \item Reduced Instruction Set Computer $\rightarrow$ Wenige, einfache Befehle
        \item Load-Store-Architektur
    \end{itemize}
\end{frame}

%%%%%%%%%%%%%%%%%%%%%%%
%% Programmübersicht %%
%%%%%%%%%%%%%%%%%%%%%%%

\begin{frame}{Programmübersicht}
    \note[item]{RISC, feste 32-Bit Befehlslänge}
    \note[item]{wenige, einfache Befehle -> geringe Codedichte, aber leicht zu übersetzten}
    \note[item]{Register-Register-Maschine, also nur Operation auf Registern in der Dreioperandenform, s. 'ERA'}
    \note[item]{Lifter 'hebt' RISC-V in die Zwischenrepräsentation (sog. IR)}
    \note[item]{Codegenerator compiliert zu Assemlby Code und dieser wird zu einer ausführbaren Datei assembliert}

    \centering
    \ProgramScheme{0.55}{13}
\end{frame}

%%%%%%%%%%%%%%%%%%%%%
%% IR Aufbau Folie %%
%%%%%%%%%%%%%%%%%%%%%

\begin{frame}{Intermediate Representation}{Aufbau}
    %\note[item]{Selbst definierte, architekturunabhängig Zwischenrepräsentation}
    %\note[item]{Hauptbestandteil: Basic Blocks}
    %\note[item]{-> Sequentielle Folge von Instruktionen, die keine Kontrollfluss ändernde Instruktion enthalten}
    %\note[item]{Basic Block enthält Variablen in sogenannter SSA-Form und die dazu gehörenden Operationen}
    %\note[item]{Eingaben kommen als "Statics" -> Abstrakte Darstellung von RISCV-Registern}
    %\note[item]{Kontrollflussoperationen verbinden die Basic Blocks}
    %\note[item]{Static single assignment -> Jeder var wird nur einmal Wert zugewiesen}
    %\note[item]{-> Vereinfacht Optimierungen}
    %\note[item]{SSA bedeutet: Variablen frei verschiebbar, deshalb Memory Tokens für Einhaltung der Reihenfolge}
    %
    %\pause

    \begin{itemize}
        \setlength{\itemsep}{1em}
        \item Programm aus \textbf{Basic Blocks}
        \item Basic Blocks enthalten Variablen
        \item Eingaben in Form von Statics
        \item Verbindung durch Kontrollflussoperationen
        \item \textbf{Static Single Assignment (SSA)} Form
    \end{itemize}
\end{frame}

\begin{frame}{Intermediate Representation}{Operationen}

    \note[item]{jump: Sprung zu basic block}
    \note[item]{ijump: Indirekter Sprung zu Addresse in Variable, mehr später}
    \note[item]{cjump: Bedingter Sprung}
    \note[item]{Aufruf-Äquivalente: call, indirekter call und zugehöriges return}

    \vspace{1cm}

    \begin{columns}[t]
        \column{0.4\textwidth}
        Instruktionen:
        \begin{itemize}
            \setlength\itemsep{0.3em}
            \item Speicher: \texttt{store}, \texttt{load}
            \item Arithmetisch: \texttt{add}, \texttt{sub}, \texttt{mul}, ...
            \item Logisch: \texttt{and}, \texttt{or}, \texttt{shl}, ...
            \item Sonstige: Typkonvertierung, ...
            \item Gleitkommaarithmetik
        \end{itemize}

        \column{0.4\textwidth}
        Kontrollflussoperationen:
        \begin{itemize}
            \setlength\itemsep{0.3em}
            \item Sprünge: \texttt{jump}, \texttt{ijump}, \texttt{cjump}
            \item Aufrufe: \texttt{call}, \texttt{icall}, \texttt{return}
            \item \texttt{unreachable}
            \item \texttt{syscall}
        \end{itemize}
    \end{columns}
\end{frame}

%%%%%%%%%%%%%%
%% IR Folie %%
%%%%%%%%%%%%%%

\begin{frame}[fragile]{Intermediate Representation}{Beispiel}
    %\pause{}
    \vspace{1.5cm}
    \begin{columns}[c]
        \column{0.25 \textwidth}
        \begin{lstlisting}[language=rv64]
[...]
addi a0, x0, 100
j somewhere_else
[...]
        \end{lstlisting}

        \column{0.1 \textwidth}
        \begin{tikzpicture}[scale=0.72]
            \TikZArrowX{TUMOrange}{0}{0}{4}{1}
        \end{tikzpicture}


        \column{0.55 \textwidth}
        \begin{lstlisting}[language=SbtIr]
block b1(/* inputs */) {
    imm v1 <- immediate 0
    imm v2 <- immediate 100
    i64 v3 <- add imm v1, imm v2
} => [(jump, [b2, i64 v3])]
    \end{lstlisting}
    \end{columns}

\end{frame}
\clearpage

\begin{frame}[fragile]{Constant Folding}{Beispiel}
    \vspace{0.3cm}
    \begin{lstlisting}[language=SbtIr]
block b1(/* inputs */) {
    imm v1 <- immediate 0
    imm v2 <- immediate 100
    i64 v3 <- add imm v1, imm v2
} => [(jump, [b2, i64 v3])]
    \end{lstlisting}

    \pause

    \vspace{1.2cm}

    \begin{lstlisting}[language=SbtIr]
block b1(/* inputs */) {
    i64 v3 <- immediate 100
} => [(jump, [b2, i64 v3])]
    \end{lstlisting}

\end{frame}
\clearpage

\begin{frame}[fragile]{Constant Folding}
    \begin{itemize}
        \setlength\itemsep{0.5em}
        \item Berechnung von Operationen, deren Eingaben bekannt sind (also Zahlen) %TODO immediate
        \item Propagieren von bekannten und berechneten Werten
        \item Vereinfachung einiger Operationen %z.b. x+0
    \end{itemize}
\end{frame}
\clearpage

\begin{frame}[fragile]{Dead Code Elimination}{Beispiel}
    \vspace{1cm}
\begin{lstlisting}[language=SbtIr]
block b1(/* inputs */) {
    i64 v3 <- immediate 100
} => [(jump, [b2, i64 v3])]

block b2(i64 v1) <= [b1] {
} => [(jump, [b3])]
\end{lstlisting}

\end{frame}

\begin{frame}[fragile]{Dead Code Elimination}
    Löschen von nicht benötigten Variablen und Entfernung von unbenutzten Parametern

    \begin{itemize}
        \setlength\itemsep{0.5em}
        \item Start bei Operationen mit Nebeneffekten, wie stores oder indirekte Jumps
        \item ``Graph-Suche'' nach allen erreichbaren Variablen
        \item Nicht erreichte Variablen sind überflüssig.
    \end{itemize}
\end{frame}
\clearpage

\begin{frame}[fragile]{Common Subexpression Elimination}{Beispiel}

\begin{lstlisting}[language=SbtIr]
block b1(i64 v1) {
    imm v2 <- immediate 100
    i64 v3 <- add i64 v1, imm v2
    imm v4 <- immediate 100
    i64 v5 <- add i64 v1, imm v4
} => [(jump, [b2, i64 v5])]
\end{lstlisting}

\pause

\vspace{1cm}

\begin{lstlisting}[language=SbtIr]
block b1(i64 v1) {
    imm v2 <- immediate 100
    i64 v3 <- add i64 v1, imm v2
} => [(jump, [b2, i64 v3])]
\end{lstlisting}

\end{frame}

\begin{frame}[fragile]{Common Subexpression Elimination}
    Löschen von redundanten Operationen und doppelten Immediates

    \begin{itemize}
        \setlength\itemsep{0.5em}
        \item Variablen von oben nach unten überprüfen
        \item Falls andere Variable mit gleichem Typ und gleichen Inputs existiert, ersetzen
    \end{itemize}
\end{frame}
\clearpage

%%%%%%%%%%%%%%%%%%%%%%%%%%%
%% ELF File Parser Folie %%
%%%%%%%%%%%%%%%%%%%%%%%%%%%

\begin{frame}[fragile]
    \frametitle{Lifter}{ELF Binärdatei laden und Instruktionsbytes decodieren}
    \begin{enumerate}
        \setlength\itemsep{0.6em}
        \item ELF File prüfen.
        \item Program Header, Sections und Symbole auslesen.
              \vspace{0.5em}
              \begin{enumerate}
                  \setlength\itemsep{0.6em}
                  \item Instruktionen aus \textit{ausführbaren} Sektionen mit \textbf{frvdec} decodieren.
                  \item Falls keine Sektionen verfügbar ist, werden Program Header verwendet
                  \item Daten aus \textit{lesbaren} Sektionen die \textbf{nicht} \textit{ausführbar} sind werden
                        byteweise als reine Daten geladen.
              \end{enumerate}
    \end{enumerate}
\end{frame}
\clearpage


%%%%%%%%%%%%%%%%%%%%%%%%%%
%% Lifter General Folie %%
%%%%%%%%%%%%%%%%%%%%%%%%%%

\begin{frame}
    \frametitle{Lifter}{RISC-V Instruktionen sequentiell in IR Code umwandeln}
    \\\vspace{0.5em}
    Wiederhole solange ungeliftete Instruktionen existieren:
    \begin{enumerate}
        \setlength{\itemsep}{0.8em}
        \item Starte neuen Basic Block bei der nächsten, ungelifteten Instruktion
        \item geladene Instruktionen in IR parsen
        \item Wiederhole solange bis:
              \begin{enumerate}
                  \vspace{1em}
                  \setlength{\itemsep}{1em}
                  \item eine \textbf{Kontrollfluss ändernde Instruktion} auftritt
                  \item der Start eines
                        neues Basic Blocks an der nächsten Instruktion registriert ist
              \end{enumerate}
        \item Beende Basic Block $\rightarrow$ markiere Start neuer Basic Blöcke an evtl. Sprungzielen
    \end{enumerate}
\end{frame}
\clearpage

%%%%%%%%%%%%%%%%%%%%%%
%% Lifter Splitting %%
%%%%%%%%%%%%%%%%%%%%%%

\begin{frame}
    \frametitle{Lifter}{Aufteilen eines Basic Blocks}
    \begin{itemize}
        \setlength{\itemsep}{1em}
        \item Situation: Sprung \textbf{in} einen bereits eingelesenen Basic Block
        \item Aufteilung an Sprungadresse in zwei Basic Blöcke
        \item Verbindung mit einem direkten Sprung
        \item Komplexe Operation, da alle Referenzen auf den geteilten Block betrachtet werden müssen
    \end{itemize}
\end{frame}
\clearpage

%%%%%%%%%%%%%%%%%%%%%%%%%%
%% Lifter Call / Return %%
%%%%%%%%%%%%%%%%%%%%%%%%%%

\begin{frame}
    \frametitle{Lifter Subroutinen Erkennung}{Erkennung der Assembler Pseudoinstruktionen \texttt{call} und \texttt{ret}.}
    \begin{itemize}
        \setlength{\itemsep}{0.8em}
        \item RISC-V Subroutinen Aufrufe funktionieren anders als bei x86\_64
        \item Standard Calling Convention definiert \texttt{x1} und \texttt{x5} als \textit{Linkregister}
        \item \texttt{JAL} := \textit{Jump and Link}
        \item \texttt{JALR} := \textit{Jump and Link Register}
        \item \texttt{JAL (x1 | x5), ..., ...} $\equiv$ \texttt{call}
        \item \texttt{JALR x0, x1, 0} $\equiv$ \texttt{ret}
        \item Laufzeitüberprüfung der Rücksprungziele durch Returnadressen-Stack
    \end{itemize}
\end{frame}
\clearpage

%%%%%%%%%%%%%%%%%%%%%%%%%%%%%%%%%
%% Lifter Jump table detection %%
%%%%%%%%%%%%%%%%%%%%%%%%%%%%%%%%%
\begin{frame}[fragile]
    \frametitle{Lifter Sprungtabellenerkennung}{Switch-Case Kontrollstrukturen werden in Sprungtabellen umgewandelt.}\vspace{1em}
    \begin{columns}[c]
        \column{0.35 \textwidth}
        \begin{lstlisting}[language=c, escapechar=!]
switch(instr.mnem) {
    case 0:
        fun0();
        break;
    case !\colorbox{yellow}{1}!:
        fun3();
        break;
    default:
        fun4();
        break;
}
        \end{lstlisting}
        \column{0.1 \textwidth}
        \begin{tikzpicture}[scale=0.72]
            \TikZArrowX{TUMOrange}{0}{0}{3}{1}
        \end{tikzpicture}
        \column{0.4 \textwidth}
        \begin{lstlisting}[language=rv64, escapechar=!]
/* [...] */
    !\colorbox{yellow}{bgtu}! a4, !\colorbox{yellow}{1}!, .L4
    slli a4, a5, 2
    lui a5, !\colorbox{yellow}{\%hi(.L1)}!
    addi a5, a5, !\colorbox{yellow}{\%lo(.L1)}!
    add a5, a4, a5
    !\colorbox{yellow}{lw}! a5, 0(a5)
    jr a5
/* [...] */
.L1:
    .word .L2
    .word .L3
        \end{lstlisting}
    \end{columns}
\end{frame}
\clearpage

%%%%%%%%%%%%%%%%%%%%%%%%%
%% Lifter Backtracking %%
%%%%%%%%%%%%%%%%%%%%%%%%%

\begin{frame}
    \frametitle{Lifter Backtracking}{Rücklverfolgung einer indirekten Sprungadresse}\\
    = rekursive Rückverfolgung der absoluten Werte aller Operationsparameter
    \begin{figure}
        \centering
        \begin{tikzpicture}[scale=1.5]
            \begin{pgfonlayer}{nodelayer}
                \node [style=BB] (0) at (-3.75, 0) {bb24};
                \node [style=VAR] (1) at (-2.25, 0) {v37};
                \node [style=OP] (2) at (-0.75, 0) {ADD};
                \node [style=VAR] (3) at (0.5, 1.5) {v36};
                \node [style=VAR] (4) at (0.5, -1.5) {v34};
                \node [style=OP] (5) at (2, -1.5) {SHL};
                \node [style=VAR] (6) at (3.5, 0.25) {v15};
                \node [style=VAR] (7) at (3.5, -1.5) {v22};
                \node [style=STATIC] (8) at (5, 0.25) {x14};
                \node [style=STATIC] (9) at (5, -1.5) {x21};
                \node [style=BB] (10) at (6.5, 1) {bb42};
                \node [style=BB] (11) at (6.5, -1) {bb23};
                \node [style=BB] (12) at (6.5, 2) {bb8};
                \node [style=BB] (13) at (6.5, 0) {bb17};
                \node [style=BB] (14) at (6.5, -2.25) {bb127};
                \node [style=VALUE] (15) at (2, 1.5) {-2};
                \node [style=VAR] (16) at (7.75, 2) {};
                \node [style=VAR] (18) at (7.75, 0) {};
                \node [style=VAR] (19) at (7.75, -1) {};
                \node [style=BB] (23) at (9.75, 1) {};
                \node [style=BB] (24) at (9.75, -2.25) {};
                \node [style=OP] (25) at (8.75, 0) {};
                \node [style=OP] (26) at (8.75, -1) {};
                \node [style=OP] (28) at (8.75, 2) {};
                \node [style=VAR] (29) at (7.75, 1) {};
                \node [style=VAR] (31) at (7.75, -2.25) {};
                \node [style=STATIC] (32) at (8.75, 1) {};
                \node [style=STATIC] (33) at (8.75, -2.25) {};
                \node [style=none] (34) at (9.75, 2) {};
                \node [style=none] (35) at (9.75, 1.5) {};
                \node [style=none] (36) at (9.75, 0.25) {};
                \node [style=none] (37) at (9.75, -0.25) {};
                \node [style=none] (38) at (9.75, -0.75) {};
                \node [style=none] (39) at (9.75, -1.25) {};
                \node [style=none] (40) at (9.75, 2.5) {};
            \end{pgfonlayer}
            \begin{pgfonlayer}{edgelayer}
                \draw [style=LINK] (0) to (1);
                \draw [style=LINK] (1) to (2);
                \draw [style=LINK] (2) to (3);
                \draw [style=LINK] (2) to (4);
                \draw [style=LINK] (3) to (15.center);
                \draw [style=LINK] (4) to (5);
                \draw [style=LINK] (5) to (7);
                \draw [style=LINK] (5) to (6);
                \draw [style=LINK] (6) to (8);
                \draw [style=LINK] (7) to (9);
                \draw [style=LINK] (9) to (14);
                \draw [style=LINK] (9) to (11);
                \draw [style=LINK] (8) to (13);
                \draw [style=LINK] (8) to (10);
                \draw [style=LINK] (8) to (12);
                \draw [style=LINK] (12) to (16);
                \draw [style=LINK] (13) to (18);
                \draw [style=LINK] (11) to (19);
                \draw [style=LINK] (16) to (28);
                \draw [style=LINK] (10) to (29);
                \draw [style=LINK] (29) to (32);
                \draw [style=LINK] (18) to (25);
                \draw [style=LINK] (14) to (31);
                \draw [style=LINK] (31) to (33);
                \draw [style=LINK] (33) to (24);
                \draw [style=LINK] (19) to (26);
                \draw [style=LINK] (32) to (23);
                \draw [style=DASHED] (28) to (34.center);
                \draw [style=DASHED] (28) to (35.center);
                \draw [style=DASHED] (25) to (36.center);
                \draw [style=DASHED] (25) to (37.center);
                \draw [style=DASHED] (26) to (38.center);
                \draw [style=DASHED] (26) to (39.center);
                \draw [style=DASHED] (28) to (40.center);
            \end{pgfonlayer}
        \end{tikzpicture}
    \end{figure}
\end{frame}
\clearpage

%%%%%%%%%%%%%%%%%%%%%
%% Generator Folie %%
%%%%%%%%%%%%%%%%%%%%%

\begin{frame}
    \frametitle{Code Generator}{Übersetzung der IR zu x86\_64}

    \note[item]{Fertige Binary enthält Assembly für Basic Blocks (Operations & CfOps), originale Binary, Helper Library}
    \note[item]{Zwei Implementierungen: Simple und Advanced}
    \note[item]{Simple Codegen optimiert wenig, dient Debugging/Development-Zwecken}
    \note[item]{Advanced Codegen allocated Register für die Operationsergebnisse, optional Merging von Operationen}
    \note[item]{Register Allocation versucht viel genutzte Werte in Registern zu halten und bei CfOps Werte in ihren momentanen Positionen zu halten}
    \note[item]{Register Allocation muss Translation Blöcke generieren!!!}
    %\note[item]{Originales ELF-Programm als binäres Speicherbild an exakter Adresse eingebunden}

    \begin{itemize}
        \setlength{\itemsep}{1em}
        \item Fertige Binary enthält Assembly für Basic Blocks, originale Binary, Stack und Runtime-Helper
        \item Zwei Implementierungen: Simple und Advanced
              %        \item Stellt nur initialen Stack zur Verfügung % es wird nicht der x86-64 Stack verwendet
        \item Simple Codegen optimiert wenig, dient Debugging/Development-Zwecken
        \item Advanced Codegen allocated Register für die Operationsergebnisse, optional Merging von Operationen
    \end{itemize}
\end{frame}
\clearpage

\begin{frame}[fragile]
    \frametitle{Beispiel}

    IR-Operation, für die Assembly generiert werden soll:
    \begin{lstlisting}[language=SbtIr]
i64 v6 <- add i64 v5, imm v1
    \end{lstlisting}

    Wenn rbx freies Register ist:
    \begin{lstlisting}[language={[x86masm]Assembler}]
lea rbx, [rax + 10]
    \end{lstlisting}

    Wenn rbx belegt und vor Variable in rax wieder benutzt:
    \begin{lstlisting}[language={[x86masm]Assembler}]
mov [rsp], rax
add rax, 10
    \end{lstlisting}
\end{frame}
\clearpage

%%%%%%%%%%%%%%%%%%%
%% IJump hashing %%
%%%%%%%%%%%%%%%%%%%

\begin{frame}
    \frametitle{IJump Lookup}{Datenstruktur für effizienten Zugriff auf Basic Block Startadressen}
    \vspace{1cm}
    \begin{columns}[t]
        \column{0.55 \textwidth}
        \textbf{Lookup Table}
        \begin{itemize}
            \vspace{1em}
            \setlength{\itemsep}{1em}
            \item Speichere für jede mögliche Jump-Addresse die korrespondierende Basic Block Startadresse
            \item Wenn diese nicht existiert, füge stattdessen 0 ein
            \item Liste dieser Wert baut Adressraum der originalen Binary nach
        \end{itemize}
        \column{0.45 \textwidth}
        \textbf{Hashing}
        \begin{itemize}
            \vspace{1em}
            \setlength{\itemsep}{1em}
            \item Ordne Startadressen in Hashtabelle ein
            \item Zweistufiges Hashing durch Einsortieren in Buckets
            \item Langsamer als Lookup Table, aber besserer Platzverbrauch
        \end{itemize}
    \end{columns}
\end{frame}
\clearpage

\begin{frame}
    \frametitle{Helper-Library}

    \begin{itemize}
        \setlength{\itemsep}{1em}
        \item Architektur-spezifische Funktionen, die für die übersetzte Binary erforderlich sind
        \item Wird zusammen mit dem Assembly des Generators gelinked
        \item Stack-Intialisierung
        \item Syscall-Translation
        \item Interpreter
    \end{itemize}
\end{frame}
\clearpage

\begin{frame}
    %\note[item]{Wird zur Auflösung von indirekten Sprüngen ohne erkanntem Sprungziel genutzt, v.a. nicht erkannte Jump Tables}
    %\note[item]{Arbeitet wie ein Emulator, also dekodiert und führt die Instruktionen zur Laufzeit aus, an gegebener und folgenden Adressen}
    %\note[item]{Von daher nötig, dass orginale Binary als Memory Image vorhanden}
    %\note[item]{Springt zu übersetztem Basic Block zurück, falls in IJump Lookup Table an der aktuellen Adresse einer gefunden wird}
    %\note[item]{Springt i.d.R. schnell zurück, v.a. ohne Translation Blocks auch häufiger \& länger nötig}
    %\note[item]{Arbeitet auf den Statics, weil er wissen muss wo die Werte der ``Register'' liegen}
    %\note[item]{Unterstützt soweit alle Standard Extensions wie der Lifter}
    %\note[item]{Kann theoretisch mit JIT Compileren und Selbstmodifizierender Code umgehen, aber ungetestet}
    %\note[item]{Kann auch alleine, d.h. ohne Lifting \& Code Generation ausgeführt werden}
    \frametitle{Interpreter}
    \vspace{1cm}
    \begin{itemize}
        \setlength{\itemsep}{1em}
        \item Dient zur Auflösung indirekter Sprünge ohne erkanntem Sprungziel
        \item Interpreteriert die RISC-V Instruktionen an gegebener Addresse
        \item Springt zurück zu übersetzem Basic Block, wenn möglich -> Ijump Lookup Table
        \item Kann theoretisch mit JIT Compileren und Selbstmodifizierender Code umgehen
    \end{itemize}
\end{frame}
\clearpage

\begin{frame}
    %\note[item]{Unterstützung für single und double precision Floating Points}
    %\note[item]{Übersetzung zu SSE und SSE2 Instruktionen, falls vorhanden auch Nutzung von FMA3 und SSE4 -> müssen erlaubt werden}
    %\note[item]{Bekanntes Problem: Unteschiede in der Implementierung führen zu leich unterschiedlichen Ergebnissen, v.a. bzgl. Rundung}
    \frametitle{Gleitkommaarithmetik}
    \begin{itemize}
        \setlength{\itemsep}{1em}
        \item Unterstützung für F und D Standard Extensions
        \item Übersetzung zu SSE und SSE2 Instruktionen
        \item Ziel: Unterstützung der Gleitkommaarithmetik, aber nicht optimiert
        \item Aber: Unterschiede in Architekturen führen zu leicht anderen Ergebnissen
        \item Beachtung des Rundungsmodus bei jeder Instruktion: zu langsam -> nur bei Konvertierung:
              \begin{itemize}
                  \item Veränderung des \textbf{MXCSR} Registers, oder
                  \item Nuzung von \textbf{roundss}/\textbf{rounsd}
              \end{itemize}
        \item Nutzung von FMA3 Instruktionen erhöht die Genauigkeit
    \end{itemize}
\end{frame}
\clearpage

\begin{frame}
    \frametitle{Rundungsprobleme}
    \centering
    \includegraphics[width=0.75\textwidth]{img/translated_diff.png}
\end{frame}

\clearpage

\begin{frame}
    \frametitle{Übersetzungprobleme bei der Gleitkommaarithmetik}
    \begin{itemize}
        \setlength{\itemsep}{1em}
        \item Einige Instruktionen können nicht direkt auf x86\_64 abgebildet werden
        \item Vorzeichenlose Ganzzahlkonvertierung: nur von AVX512 nativ unterstützt
        \item Für \textbf{Sign Injection} und \textbf{Classify} gibt es gar keine Äquivalente
        \item Diese Instruktion müssen mittels Bitarithmetik berechnet werden
    \end{itemize}
\end{frame}

\clearpage

%%%%%%%%%%%%%%%%%%%%%
%% Benchmark Setup %%
%%%%%%%%%%%%%%%%%%%%%

\begin{frame}
    \frametitle{Benchmarks}

    % Dauer: 1-2 Minuten

    \note[item]{}
    \note[item]{Wieso ist es eigentlich unmöglich native speed zu erreichen}

    \begin{itemize}
        \item SPECSpeed\textregistered2017 Integer Benchmark Suite
        \item Vergleich zu QEMU, RIA-JIT
        \item Korrektheit mittels Tests
    \end{itemize}
\end{frame}

%%%%%%%%%%%%%%%%%%%%%%%%%%%%%%%%%%%%%%%%%%%%%%%%%%%%%%%%%%%%%%%%%%%%
%% Benchmark Probleme (bzw. Optimierungen die problematisch sind) %%
%%%%%%%%%%%%%%%%%%%%%%%%%%%%%%%%%%%%%%%%%%%%%%%%%%%%%%%%%%%%%%%%%%%%

\begin{frame}
    \frametitle{Benchmarks}{Probleme}

    \begin{itemize}
        \item Hashing zu langsam
        \item Translation Blöcke in manchen Fällen wichtig
    \end{itemize}
\end{frame}

%%%%%%%%%%%%%%%%%%%%%%%
%% Benchmark Results %%
%%%%%%%%%%%%%%%%%%%%%%%

\begin{frame}
    \frametitle{Benchmarks}{Ergebnisse}

    \note[item]{}

    % TODO: insert graphik

%%========== START COPY ==========%%
\begin{figure}
    \begin{centering}
        \begin{tikzpicture}
            \begin{axis}[
                    %width=0.9\textwidth,
                    %height=0.45\textheight,
                    width=\textwidth,
                    height=0.6\textheight,
                    ybar,
                    bar width = 9pt,
                    ylabel={Execution time normalised to native % (less is better)
                    },
                    symbolic x coords = {
                            600.perlbench,
                            602.gcc,
                            605.mcf,
                            620.omnetpp,
                            623.xalancbmk,
                            625.x264,
                            631.deepsjeng,
                            641.leela,
                            648.exchange2,
                            657.xz,
                            average
                        },
                    xticklabel style = {
                            anchor = east,
                            rotate = 45 - 15 %% MODIFIED
                        },
                    xtick = data, % make a tick at every symbolic x coordinate
                    ytick distance = 1, % make a tick every 1
                    ymajorgrids = true, % display horizontal guide lines
                    legend cell align = left,
                    legend pos = north east,
                ]

                \addplot[fill=green]  table[x=benchmark, y expr=\thisrow{native} /\thisrow{native}] from {../Ausarbeitung/tikz_src/benchmarks_results1.txt};
                \addplot[fill=orange] table[x=benchmark, y expr=\thisrow{qemu}   /\thisrow{native}] from {../Ausarbeitung/tikz_src/benchmarks_results1.txt};
                \addplot[fill=yellow] table[x=benchmark, y expr=\thisrow{ria}    /\thisrow{native}] from {../Ausarbeitung/tikz_src/benchmarks_results1.txt};
                \addplot[fill=blue]   table[x=benchmark, y expr=\thisrow{sbt1}   /\thisrow{native}] from {../Ausarbeitung/tikz_src/benchmarks_results1.txt};

                \legend{
                    native,
                    QEMU,
                    RIA-JIT,
                    SBT}
            \end{axis}
        \end{tikzpicture}
    \end{centering}
\end{figure}
%%========== END COPY ==========%%

\end{frame}

%%%%%%%%%%%%%%%%%%%%%
%% Erreichte Ziele %%
%%%%%%%%%%%%%%%%%%%%%

\begin{frame}
    \frametitle{Erreichte Ziele}{Erreichte Hauptziele und Ausblick}

    % Dauer: mind. 1 Minute

    % Dauer: mind. 1 Minute, ansonsten flexible
    \note[item]{Freistehend => ohne standard bibliothek, z\.B\. handgeschriebenes assembly oder einfache C programme}
    \note[item]{Atomics werden nur also nicht atomics übersetzt so weit nötig}
    \note[item]{Demonstriert im Beispiel}
    \note[item]{2 Bugs in musl libc, malloc, fehlender BasicBlock in printf\_core wegen switch statement ohne control fluss änderung am ende von cases}
    \note[item]{Und noch vereinzelt weiter instructionen je nach bedarf}
    \note[item]{ijump eigentlich bei erweiterte Ziele, aber extrem wichtig da für return verwendet}

    \begin{itemize}
        \item Korrektes Übersetzen einfacher freistehender Programme
        \item Übersetzen von unmodifizierter musl libc
        \item Unterstützung von RV64I, RV64C, \ldots{}
        \item Auflösung indirekter Sprünge zur Laufzeit
    \end{itemize}
\end{frame}

%%%%%%%%%%%%%%%%%%%
%% Noch Fragen ? %%
%%%%%%%%%%%%%%%%%%%

\begin{frame}
    \frametitle{Noch Fragen?}{}
\end{frame}

\appendix

%%%%%%%%%%%%%%%%%%%%%%%%%
%% Generator-Translate %%
%%%%%%%%%%%%%%%%%%%%%%%%%

\begin{frame}[fragile]
    \frametitle{Generator}
    \begin{columns}[c]
        \column{0.45 \textwidth}
        \begin{lstlisting}[language=SbtIr, escapechar=!]
        block b1(/* inputs */) <= [/* predecessors */] {
            !\only<1|handout:0>{\colorbox{yellow}{i64 v0 <- @1}}!!\alt<1>{}{\colorbox{white}{i64 v0 <- @1}}!
            !\only<2|handout:0>{\colorbox{yellow}{imm v1 <- immediate 0}}!!\alt<2>{}{\colorbox{white}{imm v1 <- immediate 0}}!
            !\only<3|handout:0>{\colorbox{yellow}{imm v2 <- immediate 100}}!!\alt<3>{}{\colorbox{white}{imm v2 <- immediate 100}}!
            !\only<4|handout:0>{\colorbox{yellow}{i64 v3 <- add i64 v1, i64 v2}}!!\alt<4>{}{\colorbox{white}{i64 v3 <- add i64 v1, i64 v2}}!
        } => [!\only<5|handout:0>{\colorbox{yellow}{(jump, [b2, i64 v34])}}!!\alt<5>{}{\colorbox{white}{(jump, [b2, i64 v34])}}!]
    \end{lstlisting}

        \column{0.125 \textwidth}
        \begin{tikzpicture}[scale=0.72]
            \TikZArrowX{TUMOrange}{0}{0}{4}{1}
        \end{tikzpicture}

        \column{0.35 \textwidth}
        \begin{lstlisting}[language=rv64, escapechar=!]
b1:
!\colorbox{white}{push rbp}!
!\colorbox{white}{mov rbp, rsp}!
!\colorbox{white}{sub rsp, 32}!
!\only<1|handout:0>{\colorbox{yellow}{mov rax, [s0]}}!!\alt<1>{}{\colorbox{white}{mov rax, [s0]}}!
!\only<1|handout:0>{\colorbox{yellow}{mov [rbp - 8 * 1], rax}}!!\alt<1>{}{\colorbox{white}{mov [rbp - 8 * 1], rax}}!
!\only<2|handout:0>{\colorbox{yellow}{mov [rbp - 8 * 2], 0}}!!\alt<2>{}{\colorbox{white}{mov [rbp - 8 * 2], 0}}!
!\only<3|handout:0>{\colorbox{yellow}{mov [rbp - 8 * 3], 100}}!!\alt<3>{}{\colorbox{white}{mov [rbp - 8 * 3], 100}}!
!\only<4|handout:0>{\colorbox{yellow}{mov rax, [rbp - 8 * 2]}}!!\alt<4>{}{\colorbox{white}{mov rax, [rbp - 8 * 2]}}!
!\only<4|handout:0>{\colorbox{yellow}{mov rbx, [rbp - 8 * 3]}}!!\alt<4>{}{\colorbox{white}{mov rbx, [rbp - 8 * 3]}}!
!\only<4|handout:0>{\colorbox{yellow}{add rax, rbx}}!!\alt<4>{}{\colorbox{white}{add rax, rbx}}!
!\only<4|handout:0>{\colorbox{yellow}{mov [rbp - 8 * 4], rax}}!!\alt<4>{}{\colorbox{white}{mov [rbp - 8 * 4], rax}}!
!\only<5|handout:0>{\colorbox{yellow}{mov rax, [rbp - 8 * 4]}}!!\alt<5>{}{\colorbox{white}{mov rax, [rbp - 8 * 4]}}!
!\only<5|handout:0>{\colorbox{yellow}{mov [s0], rax}}!!\alt<5>{}{\colorbox{white}{mov [s0], rax}}!
!\only<5|handout:0>{\colorbox{yellow}{jmp b2}}!!\alt<5>{}{\colorbox{white}{jmp b2}}!
        [...]
        \end{lstlisting}

    \end{columns}

\end{frame}
\clearpage

%%%%%%%%%%%%%%%%%%%%%%%%%%%%%%%%%%%%
%% Split Basic Block Backup Folie %%
%%%%%%%%%%%%%%%%%%%%%%%%%%%%%%%%%%%%

\begin{frame}[fragile]
    \frametitle{Basic Block Splitting}{}
    \begin{columns}[c]
        \column{0.3 \textwidth}
        \begin{lstlisting}[language=rv64]
    [...]
    addi a1, x0, 100
loop:
    addi a1, a1, -1
    bne a1, x0, loop
    [...]
        \end{lstlisting}

        \column{0.05 \textwidth}
        \begin{tikzpicture}[scale=0.5]
            \TikZArrowX{TUMOrange}{0}{0}{2}{1}
        \end{tikzpicture}

        \column{0.6 \textwidth}
        \begin{lstlisting}[language=SbtIr]
block b1(inputs) <= [predecessors] {
    i64 v0 <- @1
    [...] // more statics
    imm v33 <- immediate 0
    imm v34 <- immediate 100
    i64 <- add i64 v33, i64 v33
    [...]
} => [(jump, [b2, ...])]

block b2(inputs) <= [predecessors] { // loop
    [...] // statics
    imm v33 <- immediate -1
    i64 v34 <- add i64 v11, i64 v33
    [...]
} => [(cjump, [b2, ...]), (jump, [...])]
    \end{lstlisting}
    \end{columns}

\end{frame}
\clearpage

%%%%%%%%%%%%%%%%%%%%%%%
%% Basic Block Folie %%
%%%%%%%%%%%%%%%%%%%%%%%
\begin{frame}[fragile]
    \frametitle{Lifter Basic Blöcke}{}
    \begin{columns}[c]
        \column{0.35 \textwidth}
        \begin{lstlisting}[language=c, escapechar=!]
/* [...] */
0x6: add a5, a5, a4
0x8: !\colorbox{yellow}{bltu}! a5, a4, 10
0xc: lw a5, 0(a4)
0xe: sll a4, a5, a4
/* [...] */
        \end{lstlisting}
        \column{0.1 \textwidth}
        \begin{tikzpicture}[scale=0.5]
            \TikZArrowX{TUMOrange}{0}{0}{3}{1}
        \end{tikzpicture}
        \column{0.50 \textwidth}
        \begin{lstlisting}[language=SbtIr, escapechar=!]
/* [...] */
i64 v33 <- add v13, v14
imm v34 <- immediate 0x10
imm v35 <- immediate 0xc
} => !\colorbox{yellow}{[/* cjump */]}!

!\colorbox{yellow}{block b2(/**/)}! <= [/**/] {
imm v33 <- immediate 0
i32 v34 <- load v13, v33
i64 v35 <- sll v13, v14
/* [...] */
        \end{lstlisting}
    \end{columns}
\end{frame}
\clearpage

%%%%%%%%%%%%%%%%%%%%%%%%%%%%%%
%% Technische Demonstration %%
%%%%%%%%%%%%%%%%%%%%%%%%%%%%%%

\begin{frame}[fragile]
    \frametitle{Technische Demonstration}{Live Demonstration des aktuellen Standes}

    % Dauer: ~3m
    \note[item]{Kompilieren zu RISCV-64 mit einer unmodifizierten musl libc (standardbibliothek)}
    \note[item]{Unser Program übersetzt zu x86 assembly + binary image}
    \note[item]{Assemblieren mittels GNU tools}
    \note[item]{Linken von binary image, übersetztem assembly und hilfsbibliothek}
    \note[item]{Ausführen}
    \note[item]{Ausführen, ungefährere Vergleich (native, qemu, translated) => Wo sind wir mit optimierungen, fast faktor 10 langsamer als QEMU}

    % Setup:
    % Use xfc4, configure beamer as display to the right
    % Go To Workspace 1, start pympress with
    % LC_ALL=de_DE.UTF-8 pympress presentation/build/presentation_16_9.pdf
    % Go To Workspace 2, on laptop start xfce4-terminal with
    % 1. Open new Terminal Window on laptop with `tmux new-session -s demo -A`
    % 2. Open new Terminal Window on presentation screen with `tmux attach-session -r -t demo`
    % Resize as necessary (ctrl + + or -)
    %
    % Commands:
    % ./big_tests/sysroot/bin/riscv64-linux-gnu-gcc -static examples/helloworld3.c -o examples/helloworld3
    %
    % ./build/src/translate --debug=true --print-ir=true --output-binary=examples/helloworld3.bin --output=examples/helloworld3_translated.s examples/helloworld3
    % gcc -c examples/helloworld3_translated.s -o examples/helloworld3_translated.o
    % ld -T src/generator/x86_64/helper/link.ld examples/helloworld3_translated.o build/src/generator/x86_64/helper/libhelper.a -o examples/helloworld3_translated
    %
    % ./examples/helloworld3_translated
    %
    % Additional:
    % gcc examples/helloworld3.c -o examples/helloworld3_native
    % time ./examples/helloworld3_native
    % time ./examples/helloworld3
    % time ./examples/helloworld3_translated
    \begin{columns}[c]
        \column{0.50 \textwidth}
        \begin{lstlisting}[language=bash, basicstyle=\small\ttfamily, extendedchars=true,escapeinside={@@}]
./big_tests/sysroot/bin/riscv64-linux-gnu-gcc \
    -static \
    examples/helloworld3.c \
    -o examples/helloworld3

./build/src/translate \
    @-@-debug=true \
    @-@-output-binary=examples/helloworld3.bin \
    @-@-output=examples/helloworld3_translated.s \
    examples/helloworld3
    \end{lstlisting}

        \column{0.5 \textwidth}
        \begin{lstlisting}[language=bash, basicstyle=\small\ttfamily, extendedchars=true,escapeinside={@@}]
gcc \
    -c examples/helloworld3_translated.s \
    -o examples/helloworld3_translated.o

ld \
    -T src/generator/x86_64/helper/link.ld \
    examples/helloworld3_translated.o \
    build/src/generator/x86_64/helper/libhelper.a \
    -o examples/helloworld3_translated

./examples/helloworld3_translated
    \end{lstlisting}
    \end{columns}
\end{frame}
\clearpage


%%%%%%%%%%%%%%%%%%%%%%%%%%%%%%%%%%%%%%%%%%%%%%%%%%%%%%%%%%%%%%%%%%%%%%%%%%%%%%%%
\end{document} % !!! NICHT ENTFERNEN !!!
%%%%%%%%%%%%%%%%%%%%%%%%%%%%%%%%%%%%%%%%%%%%%%%%%%%%%%%%%%%%%%%%%%%%%%%%%%%%%%%%
